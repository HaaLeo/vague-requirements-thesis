\subsection{Precision}
\label{chp:fundamentals:sec:metrics:subsec:precision}

The first metric used is \textit{precision}
It is defined \textit{"as the probability that an item is relevant given that it is detected by the algorithm"} \parencite{Zhu:2004}.
This means if an algorithm is optimized for precision, the aim is that all selected items are relevant.
However, this measure only considers the items selected by the algorithm meaning it neglects the relevant items which were not selected by the algorithm.
Precision is defined as:

\begin{equation}\label{eq:precision}
    prec = \frac{\acp{TP}}{\acp{TP}+\acp{FP}}
\end{equation}

As an example the algorithm which was used to classify the elements in \cref{fig:metrics:tp_vis} has a precision $prec = \frac{3}{3+4} = \frac{3}{7}\approx 0.43$.
