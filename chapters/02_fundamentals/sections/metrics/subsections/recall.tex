\subsection{Recall}
\label{chp:fundamentals:sec:metrics:subsec:Recall}

The next metric we wan to introduce is called \textit{recall} and is defined \textit{"as the probability of detecting an item given that it is relevant"} \parencite{Zhu:2004}.

\begin{equation}\label{eq:recall}
    rec = \frac{TP}{TP+FN}
\end{equation}

This metric expresses the performance of an algorithm as whether most of the relevant items are selected.
However, irrelevant items which may also be selected will not be penalized.
Therefore, an algorithm which simply selects \textit{all} available items will always achieve a recall $rec=1$.
For the date shown in \cref{fig:metrics:tp_vis}, the corresponding algorithm achieves a recall of $rec = \frac{3}{3+9} = 0.25$.
