\subsection{$F_1$ Score}
\label{chp:fundamentals:sec:metrics:subsec:f1_score}

We previously introduced the two metrics \textit{precision} and \textit{recall}.
From their definition in \cref{eq:precision} and \cref{eq:recall} one concludes immediately that both are only optimal if an algorithm manages to select \textit{exclusively} \ac{TP} which would lead to $prec = rec = 1$.
However, in most scenarios there do not only exist \ac{TP}.
If that is the case, it is proven that there is a trade-off among whether one wants to identify all available relevant items (recall) or all of the selected items should be relevant (precision) \parencite{Gordon:1989}.
Here we do not focus on the formal derivation of this well known trade-off, instead excellent derivations can be found in \textcites{Gordon:1989}{Zhu:2004}.
To measure and express this trade-off one can build the \textit{harmonic mean} of precision and recall which is called \textit{$F_1$ score} \parencite{Powers:2011}.
This metric is defined by the following equation:

\begin{equation}\label{eq:f1_score}
    F_1 = 2 \frac{prec \cdot rec}{prec+rec}
\end{equation}

For the example data of \cref{fig:metrics:tp_vis} its algorithm achieves an F$_1$ score of $F_1 = 2 \frac{\frac{3}{7} \cdot \frac{1}{4}}{\frac{3}{7}+\frac{1}{4}} \approx 0.32$.
The $F_1$ score's generalization for a multi-class scenario is called \textit{Macro $F_1$} score \parencite{Opitz:2019}.
