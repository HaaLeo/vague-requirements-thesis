\section{Metrics}
\label{chp:fundamentals:sec:metrics}

Usually different metrics are used to evaluate classification models.
This chapter defines required terms before the metrics are introduced in the following subchapters.

When introducing metrics for predictions it is good practice to consider the prediction labels as \textit{positives} and \textit{negatives} respectively.
Then the correctly predicted positives are referred to as \acp{TP} and the incorrectly positive classified items as \acp{FP}.
The same scheme applies to the negatives and they are therefore distinguished into \acp{TN} and \acp{FN}.
\Cref{fig:metrics:tp_vis} visualizes those groups. \parencite{Powers:2011}

In \cref{fig:metrics:tp_vis} an example algorithm retrieved three relevant items referred to as \acp{TP}.
The four items falsely classified as relevant are the \acp{FP}.
The missed relevant items are the \acp{FN}.
The algorithm falsely missed nine relevant items (\acp{FN}) in the example data of \cref{fig:metrics:tp_vis}.
The algorithm correctly did not select eleven irrelevant items (\acp{TN}).
We refer to the data shown in \cref{fig:metrics:tp_vis} when introducing the metrics in the following subsections.
\begin{figure}[htpb]
    \centering
    \def\svgwidth{\columnwidth}
    \input{figures/metrics/True_Positives.pdf_tex}
    \caption[Visualization of True Positives]{A visualization of \ac{TP}, \ac{FP}, \ac{TN} and \ac{FN}.}\label{fig:metrics:tp_vis}
\end{figure}

\subsection{Precision}
\label{chp:fundamentals:sec:metrics:subsec:precision}

\subsection{Recall}
\label{chp:fundamentals:sec:metrics:subsec:Recall}

\subsection{F1 Score}
\label{chp:fundamentals:sec:metrics:subsec:f1_score}

We previously introduced the two metrics \textit{precision} and \textit{recall}.
From their definition in \cref{eq:precision} and \cref{eq:recall} concludes immediately that both are only optimal if an algorithm manages to select \textit{exclusively} \ac{TP} which would lead to $prec = rec = 1$.
However, in most scenarios there is a trade-off among whether one wants to identify all available relevant items (recall) or all of the selected items should be relevant (precision).
Here we do not focus on the formal derivation of this well known trade-off, instead excellent derivations can be found in \textcites{Gordon:1989}{Zhu:2004}.
To measure and express this trade-off one can build the \textit{harmonic mean} of precision and recall which is called \textit{F1 score} \parencite{Powers:2011}.
This metric is defined by the following equation:

\begin{equation}\label{eq:f1_score}
    F_1 = 2 \frac{prec \cdot rec}{prec+rec}
\end{equation}

For the example data of \cref{fig:metrics:tp_vis} its algorithm achieves an F$_1$ score of $F_1 = 2 \frac{\frac{3}{7} \cdot \frac{1}{4}}{\frac{3}{7}+\frac{1}{4}}$.

\subsection{Average precision}
\label{chp:fundamentals:sec:metrics:subsec:average_precision}

