\section{Vagueness}
\label{chp:fundamentals:sec:vagueness}
In general, vagueness describes some kind of uncertainty.
However, it is quite difficult to give a precise and concise definition of vagueness.
Therefore, this chapter gives an overview how vagueness is defined across other research fields.
At the end of this chapter it is defined what a \textit{vague requirement} is.

The first discipline to look at is philosophy.
According to \textcite{Braun:2007} an expression is vague when a speaker cannot determine whether the expression applies correctly although he/her knows all relevant information.
An example is a series of different color panels which continuously vary from blue to yellow.
Although one can clearly see each patch, it is hard to distinguish whether some intermediate color panel is blue or yellow.
\textcite{Braun:2007} \textit{"think that vagueness occurs when there exist multiple equally good candidates to be the meaning of a given linguistic expression"}.
Applying their understanding of vagueness in the upper example "Blue" is vague since their exist multiple colors which are referred to as "Blue".

%Todo:
To be continued
