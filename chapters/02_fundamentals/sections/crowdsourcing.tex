\section{Crowdsourcing}
\label{chp:fundamentals:sec:crowdsourcing}
Manually labeling large datasets is a tedious task and could take weeks or even months \parencite{Welinder:2010}.
According to \textcite{Welinder:2010}, this could also include the training of people on custom interfaces and one must ensure that the annotators stay motivated in order to produce high quality annotations.
One way to overcome this limitations ist the usage of \textit{crowdsourcing}.
With crowdsourcing dozens of people can access the annotation task and contribute to the dataset.
\textcite{Howe:2008} defines crowdsourcing as \textit{"the act of taking a task traditionally performed by a designated agent (such as an employee or a contractor) and outsourcing it by making an open call to an undefined but large group of people"}.
Examples for crowdsourcing tasks are the annotation of images or the execution of surveys.
Throughout this work we follow the definition of \textcite{Howe:2008} for crowdsourcing.
