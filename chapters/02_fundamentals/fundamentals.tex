\chapter{Fundamentals}
\label{chp:fundamentals}

This chapter aims to establish the fundamental knowledge which is required to understand this thesis.
First we will introduce vagueness in the context of \ac{RE} to specify what we consider a vague requirement.
After that, we present a method named crowdsourcing to rapidly generate a labeled dataset.
To evaluate whether transformers are capable to classify vague requirements we use dedicated metrics which are explained in \cref{chp:fundamentals:sec:metrics}.
Since our chosen approach bases on state of the art \ac{ML} concepts and \acp{DNN} we explain those in \cref{chp:fundamentals:sec:machine_learning}.
As last part of the fundamentals we introduce \ac{IRA} in \cref{chp:fundamentals:sec:inter_rater_agreement} in order to judge the quality of an annotated dataset.

\section{Vagueness}
\label{chp:fundamentals:sec:vagueness}
In general, vagueness describes some kind of uncertainty.
However, it is quite difficult to give a precise and concise definition of vagueness.
Therefore, this chapter gives an overview how vagueness is defined across other research fields.
At the end of this chapter it is defined what a \textit{vague requirement} is.

The first discipline to look at is philosophy.
According to \textcite{Braun:2007} an expression is vague when a speaker cannot determine whether the expression applies correctly although he/her knows all relevant information.
An example is a series of different color panels which continuously vary from blue to yellow.
Although one can clearly see each patch, it will be hard to distinguish whether some intermediate color panel is blue or yellow.
\textcite{Braun:2007} \textit{"think that vagueness occurs when there exist multiple equally good candidates to be the meaning of a given linguistic expression"}.
Applying their understanding of vagueness in the upper example "Blue" is vague since their exist multiple colors which are referred to as "Blue".

\section{Crowdsourcing}
\label{chp:fundamentals:sec:crowdsourcing}
Manually labeling large datasets is a tedious task and could take weeks or even months \parencite{Welinder:2010}.
According to \textcite{Welinder:2010}, this could also include the training of people on custom interfaces and one must ensure that the annotators stay motivated in order to produce high quality annotations.
One way to overcome this limitations ist the usage of \textit{crowdsourcing}.
With crowdsourcing dozens of people can access the annotation task and contribute to the dataset.
\textcite{Howe:2008} defines crowdsourcing as \textit{"the act of taking a task traditionally performed by a designated agent (such as an employee or a contractor) and outsourcing it by making an open call to an undefined but large group of people"}.
Examples for crowdsourcing tasks are the annotation of images or the execution of surveys.
Throughout this work we follow the definition of \textcite{Howe:2008} for crowdsourcing.

\section{Metrics}
\label{chp:fundamentals:sec:metrics}

In order to evaluate classification models usually different metrics are used.
This chapter introduces the metrics which we use to evaluate the trained models.

\begin{figure}[htpb]
    \centering
    \def\svgwidth{\columnwidth}
    \input{figures/metrics/True_Positives.pdf_tex}
    \caption[Visualization of True Positives]{A visualization of true positives, true negatives, false positives and false negatives.}\label{fig:metrics:tp_vis}
\end{figure}


\subsection{Precision}
\label{chp:fundamentals:sec:metrics:subsec:precision}

\subsection{Recall}
\label{chp:fundamentals:sec:metrics:subsec:Recall}

\subsection{Average precision}
\label{chp:fundamentals:sec:metrics:subsec:average_precision}


\section{Machine Learning}
\label{chp:fundamentals:sec:machine_learning}
In this section we want to establish the fundamentals regarding \ac{ML} needed throughout this thesis.
We presume that the reader has basic knowledge on the domain of \ac{ML} and \acp{NN}.
If one wishes to brush up his/her knowledge on this domains we recommend \citeauthor{Alpaydin:2020}'s Introduction to Machine Learning \parencite{Alpaydin:2020}.
Instead, here we want to close the gap between basic \ac{ML} concepts and state of the art techniques.


\subsection{Transformer}
\label{chp:fundamentals:sec:machine_learning:subsec:transformer}
The \textit{transformer} was introduced in \citeyear{Vaswani:2017} by \textcite{Vaswani:2017}.
In this subsection we want to introduce the transformer and its architecture.

\subsubsection{High Level Architecture}
As its name suggests a transformer takes an input sequence and transforms it to a different output sequence.
In the context of machine translation it could take the sequence "Tom went home, because he was tired." and translate it to another language, for example to german: "Tom ging nach Hause, weil er müde war.".
The transformer consists of two main components: A stack of \textit{encoders} and a corresponding \textit{decoder} stack.
\Textcite{Vaswani:2017} use six encoders and six decoders in each stack, but the value six is arbitrary and one can use more or less en-/decoders.
\Cref{fig:fundamentals:machine_learning:transformer} visualizes the high level architecture of a transformer which consists of arbitrary many encoders and decoders.
Further, it shows an example input and the corresponding output for a machine translation task.
\begin{figure}[htpb]
    \centering
    % \def\svgwidth{\columnwidth} % Scale for height instead
    \input{figures/machine_learning/Transformer.pdf_tex}
    \caption[High Level Transformer Architecture]{The high level architecture of a transformer.}\label{fig:fundamentals:machine_learning:transformer}
\end{figure}

\subsubsection{Encoder}
In the following, we focus on the encoders, because we do not further use decoders in this work.
Before the input can be passed to the encoder stack it must be embedded, meaning that the input sequence is converted to a tensor.
This tensor is then passed through the encoder stack.
One encoder itself consists of an attention layer and and a subsequent fully connected feed forward \ac{NN} as shown in fig \cref{fig:fundamentals:machine_learning:encoder}.
% \pagebreak % Todo check page breaks
\begin{figure}[htpb]
    \centering
    % \def\svgwidth{\columnwidth} % Scale for height instead
    \input{figures/machine_learning/Encoder.pdf_tex}
    \caption[High Level Encoder Architecture]{The high level architecture of a single encoder.}\label{fig:fundamentals:machine_learning:encoder}
\end{figure}

\subsubsection{Self Attention}
The layer which enables the transformer to perform very well on a wide range of \ac{NLP} tasks is its attention layer.
\Textcite{Vaswani:2017} describe attention \textit{"as mapping a query and a set of key-value pairs to an output, where the query, keys, values, and output are all vectors"}.
The authors use the \textit{Scaled Dot-Product Attention}.
It is built by computing the dot product of all keys with all queries.
Then it is scaled by $\frac{1}{\sqrt{d_k}}$ where $d_k$ is the dimension of the keys.
According to the authors, this leads to more stable gradients.
The values' weights are obtained by applying the softmax function.
Given the query matrix $\bm{Q}$, key matrix $\bm{K}$ and a matrix $\bm{V}$ to represent the values, attention is computed by \cref{eq:attention}.


\begin{equation}\label{eq:attention}
    \text{Attention}(\bm{Q},\bm{K},\bm{V}) = \text{softmax}(\frac{\bm{Q}\bm{K}^T}{\sqrt{d_k}}) \bm{V}
\end{equation}

\subsubsection{Multi Head Attention}
According to \textcite{Vaswani:2017} it is beneficial to project all values, keys and queries $h$ times instead of computing a single attention function with $d_{model}$ dimensional values, keys, and queries with $d_{model}$ being the model's output dimension.
They perform one attention function on each of the projected versions and concatenate their outputs.
These outputs are then once more projected to obtain the final values.
The calculation of multi head attention is shown in \cref{eq:multi_head_attention}.
\begin{equation}\label{eq:multi_head_attention}
    \begin{aligned}
        \text{MultiHead}(\bm{Q},\bm{K},\bm{V}) &= \text{Concat}(head_1,\dots, head_h)\bm{W}^O\\
        \text{where }head_i &= \text{Attention}(\bm{Q}\bm{W}^Q_i, \bm{K}\bm{W}^K_i, \bm{V}\bm{W^V_i})
    \end{aligned}
\end{equation}

With $d_v$ being the dimension of the values and the parameter matrices $\bm{W}^K_i \in \mathbb{R}^{d_{model} \times d_k}$, $\bm{W}^Q_i \in \mathbb{R}^{d_{model} \times d_k}$, $\bm{W}^V_i \in \mathbb{R}^{d_{model} \times d_v}$ and $\bm{W}^O_i \in \mathbb{R}^{hd_v \times d_{model}}$.
This mechanism \textit{"allows the model to jointly attend to information from different representation subspaces at different positions"} \parencite{Vaswani:2017}.
In the example of the input sequence "Tom went home, because he was tired.", when encoding "he" multi head attention allows the model to attend to "Tom" and incorporate this information.

\subsection{Transfer Learning}
\label{chp:fundamentals:sec:machine_learning:subsec:transfer_learning}
Transfer Learning

\subsection{Local Interpretable Model-Agnostic Explanations}
\label{chp:fundamentals:sec:machine_learning:subsec:transfer_learning}
According to \textcite{Ribeiro:2016} \ac{ML} models are widely spread, although they mostly remain black boxes.
Further, they state that is essential to be able to trust a model when it used to make decisions.
However, it is very difficult for a human to trust a model which cannot explain its predictions.
To address this issue \textcite{Ribeiro:2016} introduce \ac{LIME}.

When one wants to apply \ac{LIME} for a sample $s$ to obtain an explanation for the prediction it first samples datapoints near $s$ and weights them with their distance to $s$.
Then the predictions for the sampled datapoints are generated by the original model.
This labeled dataset is used to train an interpretable linear model which approximates the original well near $s$.
Consequently, \ac{LIME} presumes that a complex model is linear on a local scale.
The explanations for the prediction of $s$ of the original model are then derived using the interpretable model. \parencite{Ribeiro:2016}


\section{Inter Rater Agreement}
\label{chp:fundamentals:sec:inter_rater_agreement}
Many \ac{ML} models require a dataset which is used to train the models.
This data is often generated by multiple rater which assign a label to each data point.
The used dataset directly influences the \ac{ML} model \parencite{Gray:2011} and therefore, well designed research studies must include mechanisms to capture \ac{IRA} \parencite{McHugh:2012}.

This chapter gives an introduction to different \ac{IRA} metrics and their applications.

\subsection{Cohen's Kappa}
\label{chp:fundamentals:sec:inter_rater_agreement:subsec:cohens_kappa}
Cohen's Kappa $\kappa$ was introduced by \textcite{Cohen:1960} in 1960.
He states the hypothesis that even if all raters are unaware of the correct answer and purely guessing, nevertheless some data points are congruent.
In his opinion that random congruency should be considered by agreement statistics.
To tackle this issue he introduced the kappa statistics to account for the random agreement among raters.
Similar to other correlation statistics can take values in the range from -1 to 1.
0 indicates the agreement obtained by random choice, whereas 1 represents perfect agreement.
The kappa calculation includes two quantities.
$p_o$ is the observed agreement of raters and $p_e$ is the probability of chance agreement.
The overall formula of Cohen's $\kappa$ is then given by \cref{eq:Cohens_kappa}:
\begin{equation}\label{eq:Cohens_kappa}
    \kappa = \frac{p_o - p_e}{1 - p_e}
\end{equation}
% TODO link to accuracy definition
$p_o$ is given by the experiments \textit{accuracy}.
For $p_e$
% next step get formula for p_e

\subsection[Scott's Pi]{Scott's $\pi$}
\label{chp:fundamentals:sec:inter_rater_agreement:subsec:scotts_pi}
\textcite{Scott:1955} developed another inter-observer agreement metric.
It was introduced specifically to measure the agreement for survey research.
This field of research includes annotating textual entities with classes by different annotators which is a common use case in \ac{NLP}.
Its aim is to measure the agreement among raters' multiple responses which are classified in exclusive categories.
According to \textcite{Scott:1955}, the percent of answers which the raters agree on as well as the \textit{consistency index S} introduced by \textcite{Bennett:1954} are biased.
Under the assumption that annotators have the same distribution of answers, he introduced his index $\pi$, referred to as \textit{Scott's $\pi$}, which has the same formula as \hyperref[chp:fundamentals:sec:inter_rater_agreement:subsec:cohens_kappa]{Cohen's kappa} (\cref{eq:Cohens_kappa}).

\begin{equation}\label{eq:Scotts_pi}
    \pi = \frac{P_o - P_e}{1 - P_e}
\end{equation}

Here $P_o$ is again the observed percentage of agreement and $P_e$ is the percentage of agreement which can be expected merely by chance.
Similar to Cohen's kappa Scott's $\pi$ is limited to two raters.
The only difference of Scott's $\pi$ to Cohen's kappa is the calculation of the by chance expected agreement $P_e$.
In contrast to \textcite{Cohen:1960} who uses the squared geometric mean of marginal proportions, \textcite{Scott:1955} used their squared arithmetic mean.
Following the notation of \cref{tab:cohens_kappa_sample_definition} this yields:

\begin{equation}\label{eq:Scotts_pi:p_e}
    P_e = \sum_{i=1}^{C} (\frac{p_{A, c_i} + p_{B, c_i}}{2})^2
\end{equation}

Applying Scott's $\pi$ to the sample data listed in \cref{tab:cohens_kappa_sample_data} one obtains the same $P_o=0.7$ but a different $P_e = (\frac{0.5 + 0.6}{2})^2 + (\frac{0.4+0.5}{2})^2 = 0.505$.
Using $P_o$ and $P_e$, yields Scott's $\pi = \frac{0.7 - 0.505}{1-0.505} = 0.\overline{39}$.

\subsection{Free-Marginal Multirater Kappa}
\label{chp:fundamentals:sec:inter_rater_agreement:subsec:free_marginal_multirater_kappa}
This is Randolph


