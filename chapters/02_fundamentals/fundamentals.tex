\chapter{Fundamentals}
\label{chp:fundamentals}

This chapter aims to establish the fundamental knowledge which is required to understand this thesis.
First we will introduce vagueness in the context of \ac{RE} to specify what we consider a vague requirement.
After that, we present a method named crowdsourcing to rapidly generate a labeled dataset.
In order to evaluate whether transformers are capable to classify vague requirements we use dedicated metrics which are explained in this chapter, too.
Since we our chosen approach bases on state of the art \ac{ML} concepts and \acp{DNN} we explain those in \cref{chp:fundamentals:sec:machine_learning}.
As last part of the fundamentals we introduce \cref{chp:fundamentals:sec:inter_rater_agreement} in order to judge the quality of an annotated dataset.

\section{Vagueness}
\label{chp:fundamentals:sec:vagueness}
In general, vagueness describes some kind of uncertainty.
However, it is quite difficult to give a precise and concise definition of vagueness.
Therefore, this chapter gives an overview how vagueness is defined across other research fields.
At the end of this chapter it is defined what a \textit{vague requirement} is.

The first discipline to look at is philosophy.
According to \textcite{Braun:2007} an expression is vague when a speaker cannot determine whether the expression applies correctly although he/her knows all relevant information.
An example is a series of different color panels which continuously vary from blue to yellow.
Although one can clearly see each patch, it will be hard to distinguish whether some intermediate color panel is blue or yellow.
\textcite{Braun:2007} \textit{"think that vagueness occurs when there exist multiple equally good candidates to be the meaning of a given linguistic expression"}.
Applying their understanding of vagueness in the upper example "Blue" is vague since their exist multiple colors which are referred to as "Blue".

%Todo:
To be continued

\section{Crowdsourcing}
\label{chp:fundamentals:sec:crowdsourcing}
Manually labeling large datasets is a tedious task and could take weeks or even months \parencite{Welinder:2010}.
According to \textcite{Welinder:2010}, this could also include the training of people on custom interfaces and additionally one must ensure that the annotators stay motivated in order to produce high quality annotations.
One way to overcome these limitations is the usage of \textit{crowdsourcing}.
With crowdsourcing many people can access the annotation task and contribute to the dataset.
\textcite{Howe:2008} defines crowdsourcing as \textit{"the act of taking a task traditionally performed by a designated agent (such as an employee or a contractor) and outsourcing it by making an open call to an undefined but large group of people"}.
Examples for crowdsourcing tasks are the annotation of images or the execution of surveys.
Throughout this work we follow the definition of \textcite{Howe:2008} for crowdsourcing.

\section{Metrics}
\label{chp:fundamentals:sec:metrics}

In order to evaluate classification models usually different metrics are used.
This chapter introduces the metrics which we use to evaluate the trained models.

\begin{figure}[htpb]
    \centering
    \def\svgwidth{\columnwidth}
    \input{figures/metrics/True_Positives.pdf_tex}
    \caption[Visualization of True Positives]{A visualization of true positives, true negatives, false positives and false negatives.}\label{fig:metrics:tp_vis}
\end{figure}


\subsection{Precision}
\label{chp:fundamentals:sec:metrics:subsec:precision}

\subsection{Recall}
\label{chp:fundamentals:sec:metrics:subsec:Recall}

\subsection{Average Precision}
\label{chp:fundamentals:sec:metrics:subsec:average_precision}

Due to the previously mentioned trade-off between recall and precision both of those metrics should be considered when optimizing a model for a task.
One metric which considers recall as well as precision is \ac{AP}.

\subsubsection{Definition}
\label{chp:fundamentals:sec:metrics:subsec:average_precision:definition}

\Textcite{Zhu:2004} defines it as the following:

\begin{equation}\label{eq:average_precision}
    AP = \sum_{i=1}^n {p(i)\Delta r(i)}
\end{equation}

Where $p(i)$ is the precision taking into account the first $i$ elements and $\Delta r(i)$ indicates the the recall change from the $i-1$th to the $i$th item.
With this metric the order of the sequence is crucial.
An algorithm which manages to sort a sequence of elements in a way that all relevant items are listed first achieves higher AP than an algorithm which performs poorly on the sorting task.

Let us consider the following example.
An algorithm was trained to return all relevant items from a dataset.
Given an unseen dataset of five items the model returns the following sorted sequence given in \cref{fig:metrics:average_precision:sample}.
The sequence is sorted in descending order, meaning the elements which the algorithm classifies as relevant with most confidence, are inserted first.
Consequently, in the example below the algorithm falsely classifies the second and fifth item as relevant.

\begin{figure}[htpb]
    \centering
    \def\svgwidth{\columnwidth}
    \input{figures/metrics/Average_Precision.pdf_tex}
    \caption[Example Sequence]{A sorted example sequence returned by some algorithm.}\label{fig:metrics:average_precision:sample}
\end{figure}

The sequence shown in \cref{fig:metrics:average_precision:sample} contains a total of three relevant items.
Following \textcite{Zhu:2004}, this yields a recall change $\Delta r(i)=\frac{1}{3}$ for a relevant item $i$ and $\Delta r(i)=0$ for an irrelevant one.
Now we can calculate the \ac{AP} according to \cref{eq:average_precision}:
\begin{equation}
    \begin{aligned}
        AP &= \sum_{i=1}^n {p(i)\Delta r(i)}\\
        &= \frac{1}{1} \cdot \frac{1}{3} + \frac{1}{2} \cdot 0 + \frac{2}{3} \cdot \frac{1}{3} + \frac{3}{4} \cdot \frac{1}{3} + \frac{3}{5} \cdot 0\\
        &=  (\frac{1}{1} + \frac{2}{3}  + \frac{3}{4}) \cdot \frac{1}{3}\\
        &\approx 0.81
    \end{aligned}
\end{equation}

\subsubsection{Interpretation}
\label{chp:fundamentals:sec:metrics:subsec:average_precision:interpretation}
The overall interpretation of \ac{AP} is intuitive: The more relevant items are ranked at the top of the returned sequence the higher is the \ac{AP} value.
However, in contrast to precision and recall, it is more difficult to judge whether the retrieved value for \ac{AP} is "good" or "bad".
Should we consider a system which yields $\ac{AP}=0.66$ as good?
In this part we want to address this issue and provide a tangible interpretation of the \ac{AP} value.

First we want to take a look at different example sequences and their corresponding \ac{AP} values.
We then derive an intuitive explanation for these example sequences which helps to judge \ac{AP} in general.
Let us consider two example sequences \textit{sequence 1} and \textit{sequence 2} shown in \cref{fig:metrics:average_precision:interpreation:sample}.

\begin{figure}[htpb]
    \centering
    \def\svgwidth{\columnwidth}
    \input{figures/metrics/Average_Precision_Example.pdf_tex}
    \caption[Two Example Sequences]{Two example sequences taken from \textcite{Tapaswi:2012}.}\label{fig:metrics:average_precision:interpreation:sample}
\end{figure}

For the first sequence we observe that \textit{every third} top ranked item is relevant whereas in the second sequence \textit{every second} item is.
We can now calculate the \ac{AP} for both sequences according to \cref{eq:average_precision} which yields $\ac{AP}_1= \frac{1}{3}$ for the first sequence and $\ac{AP}_2= \frac{1}{2}$ for the second one.
With the sequences above and their corresponding \acp{AP} $\ac{AP}_1$ and $\ac{AP}_2$ we can now conclude that an $\ac{AP}=\frac{1}{j}$ indicates that every $j$th item is relevant \parencite{Tapaswi:2012}.
With this in mind an $\ac{AP}=0.66$ is rather good, because approximately every $1.5$th item of the top ranked ones is relevant, whereas with an example $\ac{AP}=0.2$ only every fifth item is relevant.


\section{Machine Learning}
\label{chp:fundamentals:sec:machine_learning}
In this section we want to introduce the state of the art \ac{ML} concepts needed throughout this thesis.

We presume that the reader has basic knowledge in the domain of \ac{ML} and \acp{NN}.
If one wishes to brush up his or her knowledge in those domains we recommend \citeauthor{Alpaydin:2020}'s Introduction to Machine Learning \parencite{Alpaydin:2020}.
Here we only want to close the gap between basic \ac{ML} concepts and state of the art techniques.


\subsection{Transformer}
\label{chp:fundamentals:sec:machine_learning:subsec:transformer}
The \textit{transformer} was introduced in \citeyear{Vaswani:2017} by \textcite{Vaswani:2017}.
In this subsection we want to introduce the transformer and its architecture.

\subsubsection{High Level Architecture}
As its name suggests a transformer takes an input sequence and transforms it to a different output sequence.
In the context of machine translation it could take the sequence "Tom went home, because he was tired." and translate it to another language, for example german: "Tom ging nach Hause, weil er müde war.".
The transformer consists of two main components: A stack of \textit{encoders} and a corresponding \textit{decoder} stack.
\Textcite{Vaswani:2017} use six encoders and six decoders in each stack, but the value six is arbitrary and one could use more or less en-/decoders.
\Cref{fig:fundamentals:machine_learning:transformer} shows an arbitrary number of decoders and encoders which illustrates the high level architecture of a transformer.
Further, it shows an example input and the corresponding output for a machine translation task.
\begin{figure}[htpb]
    \centering
    % \def\svgwidth{\columnwidth} % Scale for height instead
    \input{figures/machine_learning/Transformer.pdf_tex}
    \caption[High Level Transformer Architecture]{The high level architecture of a transformer.}\label{fig:fundamentals:machine_learning:transformer}
\end{figure}

\subsubsection{Encoder}
In the following, we will focus on the encoders, because we will not further use decoders in this work.
Before the input can be passed to the encoder stack it must be embedded, meaning that the input sequence is converted to a tensor.
This tensor is then passed through the encoder stack.
One encoder itself consists of an attention layer and and a subsequent fully connected feed forward \ac{NN} as shown in fig \cref{fig:fundamentals:machine_learning:encoder}.
\pagebreak % Todo check page breaks
\begin{figure}[htpb]
    \centering
    % \def\svgwidth{\columnwidth} % Scale for height instead
    \input{figures/machine_learning/Encoder.pdf_tex}
    \caption[High Level Encoder Architecture]{The high level architecture of a single encoder.}\label{fig:fundamentals:machine_learning:encoder}
\end{figure}

\subsubsection{Self Attention}
The layer which enables the transformer to perform very well on a wide range of \ac{NLP} tasks is its attention layer.
\Textcite{Vaswani:2017} describe attention as \textit{"as mapping a query and a set of key-value pairs to an output, where the query, keys, values, and output are all vectors"}.
The authors use the "Scaled Dot-Product Attention".
It is built by computing the dot product of all keys with all queries.
Then it is scaled by $\frac{1}{\sqrt{d_k}}$ where $d_k$ is the dimension of the keys.
According to the authors, this leads to more stable gradients.
The weights for the values are obtained by applying the softmax function.
Given the query matrix $\bm{Q}$, key matrix $\bm{K}$ and a matrix $\bm{V}$ to represent the values, attention is computed by \cref{eq:attention}.


\begin{equation}\label{eq:attention}
    \text{Attention}(\bm{Q},\bm{K},\bm{V}) = \text{softmax}(\frac{\bm{Q}\bm{K}^T}{\sqrt{d_k}}) \bm{V}
\end{equation}

\subsubsection{Multi Head Attention}
According to \textcite{Vaswani:2017} it is beneficial to project all values, keys and queries $h$ times instead of computing a single attention function with $d_{model}$ dimensional values, keys, and queries with $d_{model}$ being the model's output dimension.
They perform one attention function on each of the projected versions and concatenate their outputs.
These outputs are then once more projected to obtain the final values.
The calculation of multi head attention is shown in \cref{eq:multi_head_attention}.
\begin{equation}\label{eq:multi_head_attention}
    \begin{aligned}
        \text{MultiHead}(\bm{Q},\bm{K},\bm{V}) &= \text{Concat}(head_1,\dots, head_h)\bm{W}^O\\
        \text{where }head_i &= \text{Attention}(\bm{Q}\bm{W}^Q_i, \bm{K}\bm{W}^K_i, \bm{V}\bm{W^V_i})
    \end{aligned}
\end{equation}

With $d_v$ being the dimension of the values and the parameter matrices $\bm{W}^K_i \in \mathbb{R}^{d_{model} \times d_k}$, $\bm{W}^Q_i \in \mathbb{R}^{d_{model} \times d_k}$, $\bm{W}^V_i \in \mathbb{R}^{d_{model} \times d_v}$ and $\bm{W}^O_i \in \mathbb{R}^{hd_v \times d_{model}}$.
This mechanism \textit{"allows the model to jointly attend to information from different representation subspaces at different positions"} \parencite{Vaswani:2017}.
In the example of the input sequence "Tom went home, because he was tired." multi head attention allows the model to attend to "Tom" and incorporate the information when it encodes "he".

\subsection{Transfer Learning}
\label{chp:fundamentals:sec:machine_learning:subsec:transfer_learning}
According to \textcite{Tan:2018}, deep learning suffers two main drawbacks.
Firstly the authors state that deep learning models depend heavily on the used training data.
The models require massive amounts of training data to learn the latent structure from the samples.
Further, the relationship of a model's size and the required training data scales linearly \parencite{Tan:2018}.

The second concern \textcite{Tan:2018} raise is insufficient training data.
This applies mainly in special domains where it is very hard to generate training samples.
For instance if a model uses a patient's data for training, it requires thousands of patients in order to be able to create an extensive dataset.

One approach to overcome these drawbacks is \textit{transfer learning}.
According to \textcite{Tan:2018} transfer learning \textit{"aims to improve the performance of [a] predictive function [...] for [a] learning task [...] by discover[ing] and transfer[ring] latent knowledge"}.
Further, they state that the dataset used to pre-train the model is often way bigger than the actual target dataset.
They distinguish among different types of transfer learning.
One is called \textit{network-based} transfer learning.
To apply this approach, one (partially) reuses a \ac{NN} which was pre-trained on a massive dataset in a specific source domain.
This approach assumes that the features which the \ac{NN} extracts are similar in the source and in the target domain.
Throughout this work, when we use the term "transfer learning" we refer to network-based transfer learning as defined by \textcite{Tan:2018}.

\subsection{Local Interpretable Model-Agnostic Explanations}
\label{chp:fundamentals:sec:machine_learning:subsec:transfer_learning}
According to \textcite{Ribeiro:2016} \ac{ML} models are widely spread, although they mostly remain black boxes.
Further, they state that is essential to be able to trust a model when it used to make decisions.
However, it is very difficult for a human to trust a model which cannot explain its predictions.
To address this issue \textcite{Ribeiro:2016} introduce \ac{LIME}.

When one wants to apply \ac{LIME} for a sample $s$ to obtain an explanation for the prediction it first samples datapoints near $s$ and weights them with their distance to $s$.
Then the predictions for the sampled datapoints are generated by the original model.
This labeled dataset is used to train an interpretable linear model which approximates the original well near $s$.
Consequently, \ac{LIME} presumes that a complex model is linear on a local scale.
The explanations for the prediction of $s$ of the original model are then derived using the interpretable model. \parencite{Ribeiro:2016}


\section{Inter Rater Agreement}
\label{chp:fundamentals:sec:inter_rater_agreement}
Many \ac{ML} models require a dataset which is used to train the models.
This data is often generated by multiple raters which assign a label to each data point.
The used dataset directly influences the \ac{ML} model \parencite{Gray:2011} and therefore, well designed research studies must include mechanisms to capture \ac{IRA} \parencite{McHugh:2012}.

This chapter gives an introduction to different \ac{IRA} metrics and their applications.

\subsection{Cohen's Kappa}
\label{chp:fundamentals:sec:inter_rater_agreement:subsec:cohens_kappa}
Cohen's Kappa $\kappa$ was introduced by \textcite{Cohen:1960} in 1960.
He states the hypothesis that even if all raters are unaware of the correct answer and purely guessing, nevertheless some data points are congruent.
In his opinion random congruency should be considered by agreement statistics.
To tackle this issue he introduced the kappa statistics to account for the random agreement among two raters.
Similar to other correlation statistics, it can take values in the range from -1 to 1.
0 indicates the agreement obtained by random choice, whereas 1 represents perfect agreement.
The kappa calculation includes two quantities.
$P_o$ is the proportion of observed agreement of raters and $P_e$ is the proportion of rating agreement expected to be obtained by chance.
The overall formula of Cohen's $\kappa$ is then given by the following equation:

\begin{equation}\label{eq:Cohens_kappa}
    \kappa = \frac{P_o - P_e}{1 - P_e}
\end{equation}

Consider two raters, A and B respectively, assigning $N$ data points to $C$ categories.
$n_{c_i, c_j}$ indicates how many data points were assigned by rater $A$ to class $c_i$ and to $c_j$ by rater $B$.
$p_{A, c_i}$ represents the proportion of assignments that were assigned to class $c_i$ by rater $A$.
An overview for $C=3$ is given in \cref{tab:cohens_kappa_sample_definition}.

\begin{table}[htpb]
    \centering
    \begin{tabular}{l|l|c|c|c|c}
        \multicolumn{2}{c}{}&\multicolumn{3}{c}{B}&\\
        \cline{3-5}
        \multicolumn{2}{c|}{}&$c_1$&$c_2$&$c_3$&\multicolumn{1}{c}{Proportion}\\
        \cline{2-5}
        \multirow{3}{*}{A}& $c_1$ & $n_{c_1, c_1}$ & $n_{c_1, c_2}$ &$n_{c_1, c_3}$& $p_{A, c_1}$\\
        \cline{2-5}
        & $c_2$ & $n_{c_2, c_1}$ & $n_{c_2, c_2}$ &$n_{c_2, c_3}$&$p_{A, c_2}$\\
        \cline{2-5}
        & $c_3$ & $n_{c_3, c_1}$ & $n_{c_3, c_2}$ &$n_{c_3, c_3}$ & $p_{A, c_3}$\\
        \cline{2-5}
        \multicolumn{1}{c}{} & \multicolumn{1}{c}{Proportion} & \multicolumn{1}{c}{$p_{B, c_1}$} & \multicolumn{1}{c}{$p_{B, c_2}$} & \multicolumn{1}{c}{$p_{B, c_3}$} & \multicolumn{1}{c}{$1$}\\
    \end{tabular}
    \caption[Cohen's Kappa notation overview]{Example confusion matrix.}\label{tab:cohens_kappa_sample_definition}
\end{table}

$P_o$ is then given by: %TODO check here for unwanted page breaks
\begin{equation}\label{eq:Cohens_kappa:p_o}
    P_o = \frac{\sum_{i=1}^{C} n_{c_i, c_i}}{N}
\end{equation}

Whereas the expected agreement by chance $P_e$ is calculated according to \cref{eq:Cohens_kappa:p_e}:
\begin{equation}\label{eq:Cohens_kappa:p_e}
    P_e = \sum_{i=1}^{C} p_{A, c_i} p_{B, c_i}
\end{equation}

An example for the case $C=2$ is illustrated in \cref{tab:cohens_kappa_sample_data}.
Here two raters rated 50 requirements as \textit{Good} or \textit{Bad}.

\begin{table}[htpb]
    \centering
    \begin{tabular}{l|l|c|c|c}
        \multicolumn{2}{c}{}&\multicolumn{2}{c}{A}&\\
        \cline{3-4}
        \multicolumn{2}{c|}{}&Good&Bad&\multicolumn{1}{c}{Proportion}\\
        \cline{2-4}
        \multirow{2}{*}{B}& Good & $20$ & $5$ & $0.5$\\
        \cline{2-4}
        & Bad & $10$ & $15$ & $0.5$\\
        \cline{2-4}
        \multicolumn{1}{c}{} & \multicolumn{1}{c}{Proportion} & \multicolumn{1}{c}{$0.6$} & \multicolumn{    1}{c}{$0.4$} & \multicolumn{1}{c}{$1$}\\
    \end{tabular}
    \caption[Cohen's Kappa sample data]{Example data of 2 raters assigning 50 data points to two categories.}\label{tab:cohens_kappa_sample_data}
\end{table}

Given \cref{tab:cohens_kappa_sample_data}, one can directly calculate $P_o = \frac{ 20 + 15 }{50} = 0.7$ and $P_e = 0.6 \cdot 0.5 + 0.4 \cdot 0.5 = 0.5$.
Now $P_o$ and $P_e$ are plugged in \cref{eq:Cohens_kappa} which yields a kappa $\kappa = \frac{0.7 - 0.5}{1 - 0.5} = 0.4$.

\subsection[Scott's Pi]{Scott's $\pi$}
\label{chp:fundamentals:sec:inter_rater_agreement:subsec:scotts_pi}
\textcite{Scott:1955} developed another inter-observer agreement metric.
It was introduced specifically to measure the agreement for survey research.
This field of research includes annotating textual entities with classes by different annotators which is a common use case in \ac{NLP}.
Its aim is to measure the agreement among raters' multiple responses which are classified in exclusive categories.
According to \textcite{Scott:1955}, the percent of answers which the raters agree on as well as the \textit{consistency index S} introduced by \textcite{Bennett:1954} are biased.
Under the assumption that annotators have the same distribution of answers, he introduced his index $\pi$, referred to as \textit{Scott's $\pi$}, which has the same formula as \hyperref[chp:fundamentals:sec:inter_rater_agreement:subsec:cohens_kappa]{Cohen's kappa} (\cref{eq:Cohens_kappa}).

\begin{equation}\label{eq:Scotts_pi}
    \pi = \frac{P_o - P_e}{1 - P_e}
\end{equation}

Here $P_o$ is again the observed percentage of agreement and $P_e$ is the percentage of agreement which can be expected merely by chance.
Similar to Cohen's kappa Scott's $\pi$ is limited to two raters.
The only difference of Scott's $\pi$ to Cohen's kappa is the calculation of the by chance expected agreement $P_e$.
In contrast to \textcite{Cohen:1960} who uses the squared geometric mean of marginal proportions, \textcite{Scott:1955} used their squared arithmetic mean.
Following the notation of \cref{tab:cohens_kappa_sample_definition} this yields:

\begin{equation}\label{eq:Scotts_pi:p_e}
    P_e = \sum_{i=1}^{C} (\frac{p_{A, c_i} + p_{B, c_i}}{2})^2
\end{equation}

Applying Scott's $\pi$ to the sample data listed in \cref{tab:cohens_kappa_sample_data} one obtains the same $P_o=0.7$ but a different $P_e = (\frac{0.5 + 0.6}{2})^2 + (\frac{0.4+0.5}{2})^2 = 0.505$.
Using $P_o$ and $P_e$, yields Scott's $\pi = \frac{0.7 - 0.505}{1-0.505} = 0.\overline{39}$.

\subsection{Fleiss' Kappa}
\label{chp:fundamentals:sec:inter_rater_agreement:subsec:fleiss_kappa}
Cohen's kappa and Scott's $\pi$ focus only the agreement between two raters.
To overcome this limitation \textcite{Fleiss:1971} introduced another kappa statistics as a generalization of Scott's $\pi$ \parencite{Scott:1955}.
Consider $N$ subjects, where each subject is assigned $n$ times to one of $C$ classes.
How many raters have assigned the $i$th subject to the $j$th class is indicated by $n_{ij}$.
$p_j$ is the proportion of all ratings which were classified to the $j$th class and is calculated as:

\begin{equation}\label{eq:fleiss_pj}
    p_j = \frac{1}{Nn}\sum_{i=1}^N n_{ij}
\end{equation}

The agreement of $n$ raters on the $i$th subject indicated by $P_i$, is the proportion of agreeing rater pairs out of the $n(n-1)$ possible pairs and given by the following:

\begin{equation}\label{eq:fleiss_Pi}
    \begin{aligned}
        P_i &= \frac{1}{n(n-1)} \sum_{j=1}^C n_{ij} (n_{ij}-1) \\
        &= \frac{1}{n(n-1)} (\sum_{j=1}^C n_{ij}^2 - n)\\
    \end{aligned}
\end{equation}

The mean of all $P_i$ then forms the overall agreement.

\begin{equation}\label{eq:fleiss_P_bar}
    \bar{P} = \frac{1}{N} \sum_{i=1}^N P_i
\end{equation}

The agreement achieved merely by chance $\bar{P_e}$ is similar to \textcite{Scott:1955} indicated by

\begin{equation}\label{eq:fleiss_P_e}
    \bar{P_e} = \sum_{j=1}^C p_j^2
\end{equation}

According to \textcite{Fleiss:1971}, the term $1-\bar{P_e}$ measures the agreement which can be achieved in extent to what would be possible by chance.
The actual agreement including the agreement by chance is represented by $\bar{P} - \bar{P_e}$.
Then the normalized kappa statistics is similar to \textcite{Cohen:1960} given by:

\begin{equation}\label{eq:fleiss_kappa}
    \kappa_{fleiss} = \frac{\bar{P}-\bar{P_e}}{1-\bar{P_e}}
\end{equation}

When applying Fleiss' kappa to the sample data of \cref{tab:cohens_kappa_sample_data} one can easily see that all $P_i$s are either $1$ when the raters agree or $0$ if one rater votes for "Good" and one for "Bad".
Considering this, \cref{eq:fleiss_P_bar} simplifies to $\bar{P} = \frac{20+15}{50} = 0.7$.
Since the expected agreement purely achieved by chance is calculated like for Scott's $\pi$, one immediately notices that for this test data the two metrics are equal $\kappa_{fleiss} = \pi = 0.\overline{39}$ like one would expect.

\subsection{Free-Marginal Multirater Kappa}
\label{chp:fundamentals:sec:inter_rater_agreement:subsec:free_marginal_multirater_kappa}

Popular kappa statistics, such as Fleiss' kappa \parencite{Fleiss:1971} are influenced by \textit{bias} and \textit{prevalence} which can cause low kappa values despite high agreement \parencite{Randolph:2005, Sim:2005} and therefore the interpretation is not straightforward.
Further, Fleiss' kappa assumes that the raters know beforehand how to distribute their votes over the possible categories.
This limitation is referred to as \textit{fixed marginals} \parencite{Brennan:1981}.
According to \textcite{Sim:2005}, prevalence influences the kappa coefficient if the proportion of agreements on one attribute differ strongly to the proportion of another attribute.
Bias is defined as the degree raters disagree on an attribute \parencite{Sim:2005}.
To tackle this drawback, \textcite{Randolph:2005} introduces a \textit{Free-Marginal Multirater Kappa} $\kappa_{free}$ which does not suffer the drawbacks of prevalence and bias.
Further, his approach allows free marginals, meaning that the raters are not restricted on how often they assign a subject to a specific class \parencite{Brennan:1981}.
\citeauthor{Randolph:2005} suggests to use his kappa metric when some marginals are not fixed.
Further, he points out that the number of categories must be chosen carefully, since each category which is theoretically not necessary, will falsely inflate the kappa's value.

\Citeauthor{Randolph:2005}'s kappa \parencite{Randolph:2005} follows the same formula like the previous kappa coefficients.
The overall observed agreement ($P_o$) subtracted the merely by chance expected agreement ($P_e$) is divided by the maximal adjusted chance agreement ($1-P_e$) yields the same formula as for the previous kappa \cref{eq:Cohens_kappa,eq:fleiss_kappa,eq:Scotts_pi}:

\begin{equation}\label{eq:Randolphs_kappa}
    \kappa_{free} = \frac{P_o - P_e}{1 - P_e}
\end{equation}

It also uses the observed agreement similar to \citeauthor{Fleiss:1971}  $P_o=\bar{P}$ (\cref{eq:fleiss_P_bar}).
However, its $P_e$ is defined as

\begin{equation}\label{eq:Randolphs_Pe}
P_e = \frac{1}{C}
\end{equation}

with $C$ being the number of different classes.
When plugging the test data from \cref{tab:cohens_kappa_sample_data} in \cref{eq:Randolphs_kappa}, \citeauthor{Fleiss:1971}' observed agreement $\bar{P} = 0.7$ can be reused.
With \citeauthor{Randolph:2005}'s $P_e = 0.5$ the overall kappa yields again $\kappa_{free} = \frac{0.7 - 0.5}{1 - 0.5} = 0.4$.


