% !TeX root = ../main.tex
% Add the above to each chapter to make compiling the PDF easier in some editors.

\chapter{Introduction}
\label{chp:introduction}

A software product is only as good as its development process \parencite{Hsia:1993}.
This process includes specifying and understanding requirements correctly as an integral part, known as requirements engineering.
Research has shown that requirements engineering is prone to faults which can cost additional time and money \parencite{Mendez:2006} and may lead to severe project delay \parencite{Femmer:2014}.
It is therefore desirable to avoid those drawbacks by recognizing misleading requirements at an early stage that faulty requirements can be reformulated in an unambiguous way before the next development step.
However, this is non trivial since often domain knowledge is required to uncover and resolve the issues \parencite{Femmer:2017}.
For example in "The software must include a service \textit{which} must be accessible via a user interface." it is unclear whether \textit{which} relates to \textit{The software} or \textit{service}.
In this case the requirement contains \textit{vague pronouns}.
Another example for a requirement defect are \textit{loopholes}.
A requirement stating that the software should be tested \textit{as far as possible} leaves the reader room for interpretation and thus is ambiguous.
The previously presented defects called \textit{vague pronouns} and \textit{loopholes} are examples for so called \textit{Requirement Smells} defined by \textcite{Femmer:2017}.
If a requirement smell is fulfilled it indicates that a requirement is of insufficient quality.
Consequently, requirement smells can be used to argue about the a requirement's quality.

Before one can reason about a requirement's quality, one must check whether requirement smells apply or not.
An approach to accomplish this are manual reviews.
According to \textcite{Salger:2013}, reviews have several drawbacks.
The review must be carried out by the relevant stakeholders and they must fully understand each requirement.
Consequently, the reviewer needs domain knowledge in order to perform the reviews which makes the review more challenging.
Furthermore, the result of a review depends on the reviewer him/herself \parencite{Zelkowitz:1983} and the reviewer can be distracted by the earlier mentioned requirement smells.
Therefore, \textcite{Femmer:2017} conclude that reviews are costly and time consuming.

Knowing reviews are costly regarding time and money, tooling which supports the requirements engineering would be beneficial for the quality assurance process.
Such tooling could support the reviewer by automatically indicating requirement smells and therefore speed up reviews.
Further, in industrial environments not only dedicated requirement smells are of interest but, more general speaking, ambiguous or vague requirements.
Such an assistant tool should be capable to process natural language since requirements are mostly formulated in natural language \parencite{Mich:2004} and indicate whether a requirement is vague or not.

\Ac{ML} achieved in recent history remarkable results for \ac{NLP} tasks \parencite{Khan:2016}.
An example to mention is Google's \ac{NN} model called BERT which showcased how \acp{NN} can improve performance in transfer learning tremendously \parencite{Devlin:2018}.
This recent success indicates that \acp{NN} have great potential in transfer learning and therefore maybe are capable to detect vague requirements.

The aim of this thesis is to further explore the capabilities of modern \acp{NN} in the context of detection of vague requirements and whether they are suitable candidates to improve the quality assurance process.
In order to improve quality assurance I make the following contributions:
\begin{enumerate}
    \item Create a dataset containing requirements and labels which indicate whether the corresponding requirement is vague or not
    \item Analyze whether and to what extent state of the art \acp{NN} are capable to classify requirements as \textit{vague} and \textit{not vague}
    \item Compare different Machine Learning approaches among each other to determine which performs best using dedicated metrics.
    \item To be defined
\end{enumerate}
