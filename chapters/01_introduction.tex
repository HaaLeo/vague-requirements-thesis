% !TeX root = ../main.tex
% Add the above to each chapter to make compiling the PDF easier in some editors.

\chapter{Introduction}
\label{chp:introduction}

Recent research found out that a software product is only as good as its development process \parencite{Hsia:1993}.
Correctly specifying and understanding requirements is an integral part of this process, known as \ac{RE}.
Research has shown that \ac{RE} is prone to faults which can cost additional time and money \parencite{Mendez:2016} and may lead to severe project delay \parencite{Femmer:2014}.
Further, the later design changes are introduced in the development process, the costlier these changes are \parencite{Folkestad:2001}.
\Cref{fig:introduction:relation} visualizes the relation of the costs of a change to the development process phase.
\begin{figure}[htpb]
    \centering
    \def\svgwidth{\columnwidth}
    \input{figures/introduction/Relation.pdf_tex}
    \caption[Relation of Development Phase to Cost per Change]{The relation of the development phase to the cost of design changes adapted from \textcite{Folkestad:2001}}\label{fig:introduction:relation}
\end{figure}

It is therefore desirable to avoid those drawbacks by recognizing misleading requirements at an early stage so that faulty requirements can be reformulated in an unambiguous way before the next development step.
However, this is not trivial since often domain knowledge is required to uncover and resolve the issues \parencite{Femmer:2017}.
For example in "The software must include a service \textit{which} must be accessible via a user interface." it is unclear whether \textit{which} relates to \textit{The software} or \textit{a service}.
In this case the requirement contains a requirement defect called \textit{vague pronouns}.
Another example for a requirement defect is that of \textit{loopholes}.
A requirement stating that the software should be tested \textit{as far as possible} leaves the reader room for interpretation and thus is ambiguous.
The previously presented defects called \textit{vague pronouns} and \textit{loopholes} are examples for so called \textit{Requirement Smells} defined by \textcite{Femmer:2017}.
Requirement smells can be used to assess a requirement's quality.
If a requirement smell is fulfilled it indicates that a requirement is of insufficient quality.

A common approach to check whether a requirement smell applies or not are manual reviews.
According to \textcite{Salger:2013}, reviews have several drawbacks.
The review must be carried out by the relevant stakeholders and they must fully understand each requirement.
Consequently, the reviewer needs domain knowledge in order to perform the reviews which makes the review more challenging.
Furthermore, the result of a review depends on the reviewer him/herself \parencite{Zelkowitz:1983} and the reviewer can be distracted by the earlier mentioned requirement smells themselves \parencite{Femmer:2017}.
Therefore, \textcite{Femmer:2017} conclude that manual reviews are costly and time consuming.

Tooling which supports the review process could consequently have the potential to save resources and further benefit the quality assurance process.
Such tooling could for example support the reviewer by automatically indicating requirement smells and therefore speed up reviews.
Further, in industrial environments not only dedicated requirement smells are of interest but, more generally speaking, ambiguous or vague requirements.
Because requirements are mostly formulated in natural language \parencite{Mich:2004} an assisting tool has to be capable of processing natural language and then assess whether a requirement is vague or not.

In recent history \Ac{ML} achieved remarkable results for complex \ac{NLP} tasks \parencite{Khan:2016}.
An important example to mention is Google's \ac{NN} model called \ac{BERT} which showcases how \acp{NN} can improve performance in \textit{transfer learning} tremendously \parencite{Devlin:2018}.
This recent success indicates that \acp{NN} have great potential in transfer learning and therefore could be capable of detecting vague requirements.
However, recent research lacks the exploration of vague sentences and in particular vague requirements.

The aim of this thesis is to further explore the capabilities of modern \acp{NN} in the context of detection of vague requirements and whether they are suitable candidates to improve the quality assurance process.
In order to improve \ac{RE} we make the following contributions:
\begin{enumerate}
    \item Since recent research endorses \acp{NN}' capabilities to solve complex tasks but has not yet explored this in the context of vague requirements, this thesis evaluates whether and to what extent state of the art \acp{NN} are capable of classifying requirements as \textit{vague} or \textit{not vague}.
    \item To successfully apply transfer learning one needs a set of labeled datapoints. However, as the time of writing no such datasets are available publicly.
        Therefore, we create a dataset containing requirements and labels which indicate whether the corresponding requirement is vague or not.
    \item It is known that recent \ac{ML} approaches perform well on \ac{NLP} tasks.
        However, they have not been compared among each other in the context of vague requirements.
        Therefore, we compare different \ac{ML} approaches among each other to determine which performs best using dedicated metrics.
\end{enumerate}

The thesis is structured as follows:
In \cref{chp:fundamentals} we establish the fundamental knowledge which is required throughout this thesis.
After that, we examine related research in the field of \ac{RE}.
The specific approach we use is introduced in \cref{chp:approach} and its sections.
To assess this approach we carry out a study which we describe in \cref{chp:study} in depth.
Threats to the study's validity are presented in \cref{chp:threats_to_validity}.
\Cref{chp:relation_to_existing_evidence} includes the comparison of our approach and its results to other existing approaches.
In \cref{chp:future_work} we derive opportunities for future research.
Finally, the conclusion of this thesis is given in \cref{chp:conclusion}.
