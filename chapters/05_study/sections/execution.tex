\section{Execution}
\label{chp:study:sec:execution}
In this section we describe how the study is executed.
Training \ac{ML} models involves the setting of hyperparameter.
Those have a direct impact on the performance of a trained model and the user must set them appropriately to optimize the learning routine and with that the model \parencite{Claesen:2015}.
We are perform a grid search to find a good parameter combination for each model.
We base the parameter grid for each model on the corresponding recommendations of the authors.
Further, we take a re-sampling strategy of our dataset into account for the grid search, because we deal with imbalanced data.
All our models are transformer based.
Transformer based models have the \textit{maximum input sequence length} as hyperparameter.
Input sequences are truncated when they exceed this limit and are padded with 0 if they are shorter than this parameter.
Because of our limited dataset we want to perform $k$-fold cross validation.
In order to determine how many folds we want to use we add this as parameter to our grid as well.
Since all models are transformer based the parameter grids are similar.
The parameter grid for \ac{BERT} and \ac{DistilBERT} is shown in \cref{tab:study:execution:paramter_grid:BERT}.
\begin{table}[htpb]
    \centering
    \begin{tabular}{ p{3.5cm} p{1.5cm} p{2cm} p{1.5cm} p{1.6cm} p{1.5cm} }
        \toprule
         Re-sampling & Folds & Learning Rate & Epochs & Max Length & Batch Size \\
        \midrule
            \begin{itemize}[noitemsep,topsep=0pt,leftmargin=15pt]
                \item {random

                down-sampling}
                \item {random

                up-sampling}
            \end{itemize}
            &\begin{itemize}[noitemsep,topsep=0pt,leftmargin=15pt]
                \item 4
                \item 8
            \end{itemize}
            & \begin{itemize}[noitemsep,topsep=0pt,leftmargin=15pt]
                \item $1\mathrm{e}{-06}$
                \item $5\mathrm{e}{-06}$
                \item $1\mathrm{e}{-05}$
                \item $5\mathrm{e}{-05}$
            \end{itemize}
            & \begin{itemize}[noitemsep,topsep=0pt,leftmargin=15pt]
                \item 1
                \item 2
                \item 3
            \end{itemize}
            & \begin{itemize}[noitemsep,topsep=0pt,leftmargin=15pt]
                \item 64
                \item 128
            \end{itemize}
            & \begin{itemize}[noitemsep,topsep=0pt,leftmargin=15pt]
                \item 16
                \item 32
            \end{itemize}\\
        \bottomrule
    \end{tabular}
    \caption[Parameter Grid for \ac{BERT} and \ac{DistilBERT}]{Parameter grid for \ac{BERT} and \ac{DistilBERT}.}\label{tab:study:execution:paramter_grid:BERT}
\end{table}

The parameter grid for \ac{ERNIE2.0} is similar, however the \textcite{Sun:2019a} recommend different learning rates.
This leads to the parameter grid for \ac{ERNIE2.0} shown in \cref{tab:study:execution:paramter_grid:ERNIE2.0}
\begin{table}[htpb]
    \centering
    \begin{tabular}{ p{3.5cm} p{1.5cm} p{2cm} p{1.5cm} p{1.6cm} p{1.5cm} }
        \toprule
         Re-sampling & Folds & Learning Rate & Epochs & Max Length & Batch Size \\
        \midrule
            \begin{itemize}[noitemsep,topsep=0pt,leftmargin=15pt]
                \item {random

                down-sampling}
                \item {random

                up-sampling}
            \end{itemize}
            &\begin{itemize}[noitemsep,topsep=0pt,leftmargin=15pt]
                \item 4
                \item 8
            \end{itemize}
            & \begin{itemize}[noitemsep,topsep=0pt,leftmargin=15pt]
                \item $2\mathrm{e}{-05}$
                \item $3\mathrm{e}{-05}$
                \item $4\mathrm{e}{-05}$
                \item $5\mathrm{e}{-05}$
            \end{itemize}
            & \begin{itemize}[noitemsep,topsep=0pt,leftmargin=15pt]
                \item 1
                \item 2
                \item 3
            \end{itemize}
            & \begin{itemize}[noitemsep,topsep=0pt,leftmargin=15pt]
                \item 64
                \item 128
            \end{itemize}
            & \begin{itemize}[noitemsep,topsep=0pt,leftmargin=15pt]
                \item 16
                \item 32
            \end{itemize}\\
        \bottomrule
    \end{tabular}
    \caption[Parameter Grid for \ac{ERNIE2.0}]{Parameter grid for \ac{ERNIE2.0}.}\label{tab:study:execution:paramter_grid:ERNIE2.0}
\end{table}

Now we perform grid search using the crowdsourced dataset $D_{crowd}$.
For that we split $D_{crowd}$ and used $90\%$ of the data for training and training evaluation whereas the remaining $10\%$ of the data are merely used for the final test of the trained models.
Finding a suitable learning rate is very challenging task, \textcite{Zeiler:2012} goes even further and states \textit{"[d]etermining a good learning rate becomes more of an art than science for many problems"}.
However, \textcite{Smith:2018} introduced the 1cycle learning rate schedule to dynamically adjust the learning rate during the training phase.This policy achieved remarkable results in their experiments.
We use this approach and set the initial learning rate to the value of the corresponding parameter grid.
Performing the grid search with the given parameter grids the parameter combinations which achieved the best $F_1$ score are listed in \cref{}.



% Second Iteration on D_{all}
eher chronologisch
