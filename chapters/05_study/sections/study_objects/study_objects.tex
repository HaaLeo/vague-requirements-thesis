\section{Study Objects}
\label{chp:study:sec:study_objects}
% Todo Remove list
% \begin{enumerate}
%     \item Alles was den Datensatz betrifft hier.
%     \item MTurk.
%     \item Kappa mit Vorversuchen und Drum und dran.
%     \item Erst genereller Plan. Dann verschiedene Iterationen (chronologisch).
%     \item Resultat → der fertige Datensatz.
%     \item Hier die "zweistufigen" Datensatz anführen. Erst nur MTurk, dann Mturk und manuellen.
% \end{enumerate}


% Introduction paragraph
Training and evaluation of our transformer based models requires a labeled dataset.
The dataset must consist of requirements which are labeled as vague or not vague.
However, to the extent of our knowledge there does not exist a publicly available dataset which fulfills this requirement.
Therefore, it is inevitable to create a suitable dataset ourselves.
In the following subsections we first describe how we \textit{create} the dataset and then we present its overall \textit{properties}.

\subsection{Dataset Creation}
\label{chp:study:sec:study_objects:dataset_creation}

Although there does not exist a dataset which satisfies our needs, we can fall back to a dataset which consists of 2776 unique requirements \parencite{Kummeth:2020} and base our dataset upon it.
In the next two subsections we present \textit{how} we create \textit{which} datasets and use throughout this thesis.

\section{Crowdsourcing}
\label{chp:fundamentals:sec:crowdsourcing}
Manually labeling large datasets is a tedious task and could take weeks or even months \parencite{Welinder:2010}.
According to \textcite{Welinder:2010}, this could also include the training of people on custom interfaces and one must ensure that the annotators stay motivated in order to produce high quality annotations.
One way to overcome this limitations ist the usage of \textit{crowdsourcing}.
With crowdsourcing dozens of people can access the annotation task and contribute to the dataset.
\textcite{Howe:2008} defines crowdsourcing as \textit{"the act of taking a task traditionally performed by a designated agent (such as an employee or a contractor) and outsourcing it by making an open call to an undefined but large group of people"}.
Examples for crowdsourcing tasks are the annotation of images or the execution of surveys.
Throughout this work we follow the definition of \textcite{Howe:2008} for crowdsourcing.

\subsubsection{Manual Labeling}
\label{chp:study:sec:study_objects:dataset_creation:manual_labeling}

% Paragraph on How can datasets be created. What is Amazon Mechanical Turk.

% Paragraph on prechecks which basis dataset?

% Paragraph on how did we create our MTurk dataset

\subsection{Dataset Properties}
\label{chp:study:sec:study_objects:dataset_properties}

