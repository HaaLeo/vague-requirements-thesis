\section{Interpretation}
\label{chp:study:sec:interpretation}

% Introduction paragraph. why are results bad
When we compare our results from \cref{chp:study:sec:results} with more \hyperref[chp:related_research:sec:rule_based approaches]{\textit{traditional approaches}}, one immediately observes that our models perform rather poorly.
One example is Smella, developed by \textcite{Femmer:2017}.
It uncovers \textit{requirement smells} with a precision of $59\%$ and a $82\%$ whereas our \ac{BERT} based model achieves same recall, but only a precision of $0.36$.
In contrast to our \textit{vague requirements} Smella's objective is to identify requirement smells.
Although these objectives are not identical, they are very similar.
Therefore, we can follow that Smella as an example and rule based approaches in general are severely better in identifying poor requirements.
In this section we want to reflect on the previous results presented and discuss possible causes.

% Paragraph for LIME
First we want to check which word of a sentence contributes most to the model's classification result.
To do this we use \ac{LIME}.
After that we count the occurrences of those words in $D_{all_{train}}$ and try to derive \textit{why} our models classify a majority of the dataset wrongly.
With this procedure we analyze the top four \ac{TP}, \ac{TN}, \ac{FP} and \ac{FN} of a model.
An example for a false positive analyzed by \ac{LIME} is given in \cref{fig:study:interpretation:LIME}.
The more a word's background is highlighted in green, the more it contributes to the overall prediction, in case of \cref{fig:study:interpretation:LIME} the prediction is \textit{vague}.
\begin{figure}[htpb]
    \centering
    \def\svgwidth{\columnwidth}
    \scalebox{0.80}{\subsection{Local Interpretable Model-Agnostic Explanations}
\label{chp:fundamentals:sec:machine_learning:subsec:transfer_learning}
According to \textcite{Ribeiro:2016} \ac{ML} models are widely spread, although they mostly remain black boxes.
Further, they state that is essential to be able to trust a model when it used to make decisions.
However, it is very difficult for a human to trust a model which cannot explain its predictions.
To address this issue \textcite{Ribeiro:2016} introduce \ac{LIME}.

When one wants to apply \ac{LIME} for a sample $s$ to obtain an explanation for the prediction it first samples datapoints near $s$ and weights them with their distance to $s$.
Then the predictions for the sampled datapoints are generated by the original model.
This labeled dataset is used to train an interpretable linear model which approximates the original well near $s$.
Consequently, \ac{LIME} presumes that a complex model is linear on a local scale.
The explanations for the prediction of $s$ of the original model are then derived using the interpretable model. \parencite{Ribeiro:2016}
}
    \caption[Study Interpretation: Example for LIME]{LIME scores for a false positive.}\label{fig:study:interpretation:LIME}
\end{figure}

\newpage % Todo check page break
After we we determined the occurrences of the requirements' words in the dataset we could not identify a pattern among how often a word occurs in the dataset and how it influences the prediction.
How often the words of above figure occur in vague and not-vague requirements in $D_{all_{train}}$ is shown in \cref{tab:study:interpretation:LIME}.
\begin{table}[htpb]
    \centering
    \begin{tabular}{l l l l l }
        \toprule
         Word & \ac{LIME} Score & Count in Vague Req. & Count in Not Vague Req. \\
        \midrule
        analyzes & 0.164 & 6 & 3 \\
        remaining & 0.201 & 1 & 3  \\
        life &  0.149 & 13 & 14\\
        inspection & 0.05 & 3 & 14 \\
        intervals & 0.063 & 2 & 2 \\
        \bottomrule
    \end{tabular}
    \caption[Study Interpretation: Word Occurrences]{The occurrences of a requirement's words in $D_{all_{train}}$.}\label{tab:study:interpretation:LIME}
\end{table}
% Paragraph for Architecture/ Self Attention

% Paragraph on why we do not use thresholds to "enhance" results.
