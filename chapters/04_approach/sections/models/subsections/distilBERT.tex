\subsection{DistilBERT}
\label{chp:approach:sec:models:subsec:distilbert}

\Ac{BERT} consists of several million trainable parameters, at least of 110 million to be precise.
Parameter-heavy deep learning models like \ac{BERT} tend to raise scaling issues the more parameters they include \parencite{Schwartz:2019}.
To tackle this concern, \textcite{Sanh:2019} introduced \acl{DistilBERT}.

% Paragraph architecture and learning
The architecture of \ac{DistilBERT} is similar to \ac{BERT}.
However, \ac{DistilBERT} uses only half as much layers.
To obtain a much smaller model the authors use \textit{knowledge distillation} as proposed by \textcites{Bucilua:2006}{Hinton:2015}.
When applying this technique a smaller student model is trained to replicate the behaviour of the larger model, the teacher.
The "teacher" can also be a set of different models.
With this technique the authors successfully trained \ac{DistilBERT} to keep 97\% of \ac{BERT}'s performance.
Compared to \ac{BERT}, \ac{DistilBERT} only consists of 40\% less parameters and trains 60\% faster. \parencite{Sanh:2019}

Since \ac{DistilBERT} and \ac{BERT} share the overall same architecture, \ac{DistilBERT} also exposes a [CLS] which again can be seen as an aggregation of the input sequence.
Similar to our \ac{BERT}-based model, we will again use this token as input for a fully connected feed forward \ac{NN} with 1,500 trainable parameters.
