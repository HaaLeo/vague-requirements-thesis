\subsection{BERT}
\label{chp:approach:sec:models:subsec:BERT}

% Paragraph on architecture
\Ac{BERT} is based on \textit{encoder} of the original transformer model of \textcite{Vaswani:2017}.
It is simply a stack of encoders.
There exist two variants of \ac{BERT}: \ac{BERT}$_{base}$ and \ac{BERT}$_{large}$.
The first consists of 12 encoder layers, second of 24 which yields to 110 million parameters and 310 million respectively.
Since the 24 layer variant takes severely longer to train we use the 12 encoder layer variant in this approach. \parencite{Devlin:2018}

% Paragraph on in- / outputs
\Ac{BERT} is suitable for different kinds of downstream tasks, since it is capable to embed also sentence pairs in a single sequence.
This is achieved by introducing special input tokens which have a designated meaning.
For our approach most important special token is the \textit{[CLS]} token, because its corresponding final hidden state represents the aggregation of the input sequence and can be used for further downstream classification tasks.
\Ac{BERT}'s input is the summation of three different embeddings.
The first embedding is the \textit{token embedding}.
Here each input token (for example each word of a sentence) is embedded using WordPiece embeddings \parencite{Wu:2016} which utilize a vocabulary of 30.000 words.
The second embedding is called \textit{segment embedding} and indicates to which input sequence a token belongs.
The third embedding embeds a tokens position.
The input of \ac{BERT} is obtained by summing up those three embeddings for each input token.
% Todo Figure for visualization of embedding creation here?

% Paragraph on downstream classification
Once we embed our input sentences and pass them to \ac{BERT}, \ac{BERT} aggregates the sequence in the output of the [CLS] token.
However, this raw token does not yet indicate whether a requirement is vague or not.
In order to finally classify a requirement, the [CLS] token is passed to a single layer feed forward \ac{NN}.
The output of this layer then indicates in percent how certain the model is whether a requirement is vague or not.
