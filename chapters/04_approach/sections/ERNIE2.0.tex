\section{ERNIE 2.0}
\label{chp:approach:sec:ernie2.0}

According to \textcite{Sun:2019a} the aim of most pre-training routines are based on the co-occurrence of sentences or words.
Further, they argue that with such a pre-training procedure other valuable information like lexical or semantic information is not considered.
To solve this issue, they proposed a new learning framework called \ac{ERNIE2.0}.
This framework enables a model to learn multiple tasks without forgetting the the knowledge learned by the previous tasks.
To accomplish this, their continual multi-task learning method integrates the new task by initializing a model with the previously learned parameters and then trains for the newly added task together with the prior tasks.

They pre-train a model, called \ac{ERNIE2.0} model, using the \ac{ERNIE2.0} framework to showcase that their approach works .
Like \ac{BERT} and \ac{DistilBERT} the model is based on the transformer architecture of \textcite{Vaswani:2017} and consists of multiple encoder layers.
To ease the comparison with \ac{BERT} they decide to use the same architecture but pre-train it with the \ac{ERNIE2.0} framework.
However, the input of the \ac{ERNIE2.0} model is slightly different.
In addition to the position, segment and token embedding they use another \textit{task embedding}.
The purpose of this new embedding is \textit{"to represent the characteristic of different tasks"} \parencite{Sun:2019a}.
They define word-aware and semantic aware pre-training tasks to capture lexical and syntactic information during the training phase.
The exact explanation of those tasks is not in the scope of this thesis. \parencite{Sun:2019a}

For our classification task we use the \ac{ERNIE2.0} model which was pre-trained using the \ac{ERNIE2.0} framework.
Due to the similar architecture to \ac{BERT} the \ac{ERNIE2.0} model again has a [CLS] output token which can be used for downstream classification tasks.
Analogically to the prior two models, we use a single layer feed forward \ac{NN} with 1,538 trainable parameters to classify requirements as vague or not.
