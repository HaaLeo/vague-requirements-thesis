\chapter{Approach}
\label{chp:approach}
% (wie werde ich die Vagheit detektieren → NN, Transformer Architektur) welche
% Grob vorstellen was ich machen werde:

% ich benutze mehrere NNs um Vagheit zu klassifizieren
% Eingaben, Ausgaben
% eher top Level

In this chapter the concrete approach is presented which was used to detect vague requirements.
To classify requirements as vague or not \acp{NN} are used.
The result of each \ac{NN} will be evaluated using dedicated metrics.

Transformer based approaches have turned out to be viable alternatives to traditional architectures based on \acp{RNN} for \ac{NLP} tasks like language translation \parencites{Gehring:2017}{Vaswani:2017}.
This showcases that transformers are capable to solve complex \ac{NLP} tasks.
Therefore, we also leverage a transformer based approach to attempt to classify vague requirements.
In detail, our approach considers three different models.
Namely \ac{BERT} \parencite{Devlin:2018}, \ac{DistilBERT} \parencite{Sanh:2019} and \ac{ERNIE2.0} \parencite{Sun:2019a}.
The following sub chapters outline which particular models and metrics are used.

\subsection{BERT}
\label{chp:approach:sec:models:subsec:BERT}

% Paragraph on architecture
\Ac{BERT} is based on \textit{encoder} of the original transformer model of \textcite{Vaswani:2017}.
It is simply a stack of encoders.
There exist two variants of \ac{BERT}: \ac{BERT}$_{base}$ and \ac{BERT}$_{large}$.
The first consists of 12 encoder layers, second of 24 which yields to 110 million parameters and 310 million respectively.
Since the 24 layer variant takes severely longer to train we use the 12 encoder layer variant in this approach. \parencite{Devlin:2018}

% Paragraph on in- / outputs
\Ac{BERT} is suitable for different kinds of downstream tasks, since it is capable to embed also sentence pairs in a single sequence.
This is achieved by introducing special input tokens which have a designated meaning.
For our approach most important special token is the \textit{[CLS]} token, because its corresponding final hidden state represents the aggregation of the input sequence and can be used for further downstream classification tasks.
\Ac{BERT}'s input is the summation of three different embeddings.
The first embedding is the \textit{token embedding}.
Here each input token (for example each word of a sentence) is embedded using WordPiece embeddings \parencite{Wu:2016} which utilize a vocabulary of 30,000 words.
The second embedding is called \textit{segment embedding} and indicates to which input sequence a token belongs.
The third embedding embeds a tokens position.
The input of \ac{BERT} is obtained by summing up those three embeddings for each input token.
% Todo Figure for visualization of embedding creation here?

% Paragraph on downstream classification
Once we embed our input sentences and pass them to \ac{BERT}, \ac{BERT} aggregates the sequence in the output of the [CLS] token.
However, this raw token does not yet indicate whether a requirement is vague or not.
In order to finally classify a requirement, the [CLS] token is passed to a single layer feed forward \ac{NN}.
The output of this layer then indicates in percent how certain the model is whether a requirement is vague or not.
It consists of 1,500 parameters.

\section{DistilBERT}
\label{chp:approach:sec:distilbert}

Parameter-heavy deep learning models like \ac{BERT} tend to raise scaling issues the more parameters they include \parencite{Schwartz:2019}.
To tackle this concern, \textcite{Sanh:2019} introduced \acl{DistilBERT}.

% Paragraph architecture and learning
The architecture of \ac{DistilBERT} is similar to that of \ac{BERT}.
However, \ac{DistilBERT} uses only half as many layers.
To obtain a much smaller model the authors use \textit{knowledge distillation} as proposed by \textcites{Bucilua:2006}{Hinton:2015}.
When applying this technique a smaller student model is trained to replicate the behaviour of the larger model, the teacher.
The "teacher" can also be a set of different models.
With this technique the authors successfully trained \ac{DistilBERT} to keep 97\% of \ac{BERT}'s performance.
Compared to \ac{BERT}, \ac{DistilBERT} consists of 40\% less parameters and trains 60\% faster. \parencite{Sanh:2019}

Since \ac{DistilBERT} and \ac{BERT} share the overall same architecture, \ac{DistilBERT} also exposes a [CLS] token which again can be seen as an aggregation of the input sequence.
Similar to our \ac{BERT}-based model, we use this token as input for a fully connected feed forward \ac{NN} with 1,538 trainable parameters.

\subsection{ERNIE 2.0}
\label{chp:approach:sec:models:subsec:ernie2.0}

According to \textcite{Sun:2019a} the aim of most pre-training routines are based on the co-occurrence of sentences or words.
Further, they argue that with such a pre-training procedure other valuable information like lexical or semantic information is not considered.
To solve this issue, they proposed a new learning framework called \ac{ERNIE2.0}.
This framework enables a model to learn multiple tasks without forgetting the the knowledge learned by the previous tasks.
To accomplish this, their continual multi-task learning method integrates the new task by initializing a model with the previously learned parameters and then trains for the newly added task together with the prior tasks.

To showcase that their approach works they pre-train a model, called \ac{ERNIE2.0} model, using the \ac{ERNIE2.0} framework.
Like \ac{BERT} and \ac{DistilBERT} the model is based on the transformer architecture of \textcite{Vaswani:2017} and consists of multiple encoder layer.
To ease the comparison with \ac{BERT} they decide to use the same architecture but pre-train it with the \ac{ERNIE2.0} framework.
However, the input of the \ac{ERNIE2.0} model is slightly different.
In addition to the position, segment and token embedding they use another \textit{task embedding}.
The purpose of this new embedding is \textit{"to represent the characteristic of different tasks"} \parencite{Sun:2019a}.
To capture lexical and syntactic information during the training phase they define word-aware and semantic aware pre-training tasks.
The exact explanation of those tasks is not in the scope of this thesis. \parencite{Sun:2019a}

For our classification task we will use the \ac{ERNIE2.0} model which was pre-trained using the \ac{ERNIE2.0} framework.
Due to the similar architecture to \ac{BERT} the \ac{ERNIE2.0} model again has a [CLS] output token which can be used for downstream classification tasks.
Analogical to the prior two models we will use a single layer feed forward \ac{NN} with 1,500 trainable parameters to classify requirements as vague or not.

