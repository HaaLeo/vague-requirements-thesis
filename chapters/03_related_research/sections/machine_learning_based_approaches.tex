\section{Machine Learning Based Approaches}
\label{chp:related_research:sec:machine_learingn_based_approaches}

Next to solely rule-based approaches, there exist \ac{ML} based approaches which use \ac{ML} techniques to classify a requirement.
In this chapter \ac{ML} based approaches for requirement classification are presented.

\textcite{Medlock:2007} present an approach which examines hedge classification in scientific literature.
The subject of hedge classification is to detect speculative language.
First they introduce a probabilistic model to gather training data.
The same model is used to derive a weakly supervised learner which classifies a sample as speculative when its probability exceeds a predefined threshold.
This iterative learning approach is capable to perform hedge classification with similar accuracy as other approaches based on \acp{SVM}.
They conclude that their simple \ac{ML} based approach is capable to perform hedge classification and more complex \ac{ML} approaches are likely to perform even better.

\textcite{Ormandjieva:2007} present an approach which utilizes decision trees for the classification task.
Their quality model distinguishes among \textit{surface understanding} and \textit{conceptual understanding}.
Surface understanding means the literal meaning like how difficult or easy it is to understand a requirements documents' passages.
Whereas conceptual understanding aims for a passage's interpretation.
For example how difficult it is for a developer to implement a system by only reading the paragraph.
The dataset was obtained by manual reviews of four annotators.
Each annotator carefully read a set of requirements documents and classified its passages once with respect to surface understanding and once with respect to conceptual understanding.
The inter-rator agreement is indicated by Cohen's Kappa \parencite{Cohen:1960} and is 0.64 for conceptual understanding and 0.66 for surface understanding.
For the classification task each sentence of the requirements documents was \ac{POS} tagged and its syntax parsed using the Stanford Parser \parencite{Klein:2002}.
Then the number of occurrences of the indicators for little surface understanding were counted.
Based on that information a decision tree was constructed.
The authors mention that a \ac{NN} could maybe achieve better results.
However, due to the lack of data a decision tree is preferred.
The trained decision tree was capable to solve the classification task with 86.67\% accuracy.
\textcite{Ormandjieva:2007} conclude that it is indeed feasible to uncover faults related to surface understanding using a decision tree and see their quality model approved as suitable.

\textcite{Yang:2012} use \ac{CRF} to extract uncertainty cues from requirements documents.
They manually labeled several requirements documents for uncertainty cues and their corresponding scopes.
The detection problem was formulated as a labelling task of a sequence on token-level.
Each word of a sentence is assigned a class label indicating whether the word is the first word of the cue, inside a cue or not in a cue.
To additionally gather further information regarding the semantics of a cue keyword, more linguistic features were extracted.
Examples are the word lemma, \ac{POS} tag, \ac{POS} tags of the three neighboring words and grammatical relations.
Those features and the class label are used to train a classification model using the \ac{CRF} algorithm \parencite{Lafferty:2001}.
Further, a post-processing step is applied to to capture infrequent cues extracted from the training dataset.
In this step the cues are detected using string matching.
When a match occurs the sentence is classified as speculative.
In the conducted study multiple models with a different amount of features was trained and tested.
Speculative sentences were detected with 85.58\% precision and 77.65\% recall considering all features.
Regarding uncertainty scope identification the presented system performs worse: 54.37\% precision and 49.95\% recall were achieved.
They conclude that their implemented approach works well on uncovering speculative sentences and suggest that different supervised \ac{ML} techniques could be promising in solving this task.

\textcite{Parra:2015} use \ac{ML} techniques to classify requirement regarding their quality.
The approach of choice is rule induction.
The authors use this technique because of its robustness against noise due to insufficient data.
To generate the models they use the PART rule induction algorithm \parencite{Eibe:1998}.
The obtained models use a requirement's different metrics to classify it.
The models are used to induce rules which then can be used to evaluate a requirement's quality according to the corresponding metric.
The models were examined using a dataset annotated by experts and achieved accuracy in a range between 83.27\% and 87.72\%.
The achieved recall remains unknown.
The authors conclude that it is possible to learn the quality features of a requirement and use the obtained models for quality prediction of new requirements.
They endorse to enhance the model creation process to further improve the performance.

\textcite{Stajner:2017} examine linguistic hedging in the monetary political domain.
They create two different datasets.
One includes transcribed \textit{debates} of different meetings of several committees.
The second dataset includes the meetings statement reports which are typically conducted at the end of a meeting and further referred to as \textit{decisions}.
The datasets were labeled \textit{speculative} or \textit{non-speculative} by three annotators.
The averaged pairwise Cohen's Kappa \parencite{Cohen:1960} was 0.56 for the debates and 0.61 for the decisions dataset.
The annotators marked phrases which indicate a speculative sentence in their opinion.
For the classification task a \ac{SVM} was trained on both datasets.
They used \ac{BoW}, the lists of speculation triggers or different combinations as input features for the \acp{SVM}.
The authors benchmarked their \ac{SVM} approach with a \ac{CNN} and the best performing system for the CoNLL-2010 shared task \parencite{Farkas:2010}.
The conclude that a general domain dataset can be used to train a model for a hedge classification task in this domain.
Further they showed that their \ac{SVM} approach including lists of speculation triggers performs as well as other state of the art systems in other domains and could outperform the \ac{CNN} classifier on the debates dataset.

All of the presented works use \ac{ML} techniques in order to improve their results.
Some try to build classifier which are capable to predict a new requirement's quality.
Others use those techniques to extract specific cues and their location.
However, none of the mentioned approaches examines how state of the art \acp{NN}, like \acp{DNN}, perform at classifying requirements as vague or not.
With this thesis I want to answer the question whether \acp{DNN} are promising in classifying requirements regarding their quality.
