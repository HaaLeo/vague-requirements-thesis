\section{Machine Learning Based Approaches}
\label{chp:related_research:sec:machine_learingn_based_approaches}

Next to solely rule-based approaches, there exist \ac{ML} based approaches which use \ac{ML} techniques to classify a requirement.
In this chapter \ac{ML} based approaches for requirement classification are presented.

\textcite{Ormandjieva:2007} present an approach which utilizes decision trees for the classification task.
Their quality model distinguishes among \textit{surface understanding} and \textit{conceptual understanding}.
Surface understanding means the literal meaning like how difficult or easy it is to understand a requirements documents' passages.
Whereas conceptual understanding aims for a passage's interpretation.
For example how difficult it is for a developer to implement a system by only reading the paragraph.
The dataset was obtained by manual reviews of four annotators.
Each annotator carefully read a set of requirements documents and classified its passages once with respect to surface understanding and once with respect to conceptual understanding.
The inter-rator agreement is indicated by Cohen's Kappa \parencite{Cohen:1960} and is 0.64 for conceptual understanding and 0.66 for surface understanding.
For the classification task each sentence of the requirements documents was \ac{POS} tagged and its syntax parsed using the Stanford Parser \parencite{Klein:2002}.
Then the number of occurrences of the indicators for little surface understanding were counted.
Based on that information a decision tree was constructed.
The authors mention that a \ac{NN} could maybe achieve better results.
However, due to the lack of data a decision tree is preferred.
The trained decision tree was capable to solve the classification task with 86.67\% accuracy.
\textcite{Ormandjieva:2007} conclude that it is indeed feasible to uncover faults related to surface understanding using a decision tree and see their quality model approved as suitable.
