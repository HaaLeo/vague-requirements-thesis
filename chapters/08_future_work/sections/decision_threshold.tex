\section{Decision Threshold}
\label{chp:future_work:sec:decision_threshold}
% Paragraph on why we do not use thresholds to "enhance" results.
Adjusting the \textit{decision threshold} which is by default $0.5$ is another possibility to take a dataset's imbalance into account and increase a model's performance.
It can be challenging to choose an optimal threshold because a \ac{NN} allows to set the threshold arbitrarily which means many values are a valid choice \parencite{Mazurowski:2008}.
According to \textcite{Mazurowski:2008}, it is even impossible to directly compare two systems which use different thresholds.
Further, the optimal threshold is coupled to the problem itself and depends on the objective to optimize for \parencite{Brown:2019}.
For example, when classifying safety critical requirements which could endanger lives, one would choose a more conservative threshold than when classifying less critical requirements.
The scope of this thesis is \textit{not} to optimize a requirement classification system to a specific objective but to evaluate whether a transformer-based system is generally capable of classifying the requirements.
Therefore, we do not further consider the adjustment of thresholds in this thesis and leave this topic for future research.
