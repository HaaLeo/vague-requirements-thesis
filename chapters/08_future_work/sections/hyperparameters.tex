\section{Hyperparameters}
\label{chp:future_work:sec:hyperparameters}
Finding good hyperparameters for a model is a challenging task and can limit a model's performance \parencite{Bergstra:2011}{Zeiler:2012}.
In the following two subsections we want to show two enhancements for future research which can lead to better performing algorithms.

\subsection{Re-Sampling Strategy}
\label{chp:future_work:sec:hyperparameters:re_sampling_strategy}
The dataset we used is very imbalanced.
Only a fifth of its datapoints are labeled as vague.
Therefore, we considered two re-sampling strategies in the grid-search, namely random up-sampling and random down-sampling.
Those techniques take account for a datasets imbalance by randomly duplicating entries of minority class or discarding samples of the majority class.

However, there also exist more sophisticated re-sampling techniques.
One example is \ac{SMOTE} \parencite{Chawla:2002}.
Instead of randomly duplicating entries of the minority class, it generates new sample of the minority class which are "close" to the other datapoints.
For an extensive explanation of \ac{SMOTE} we refer to the original of \textcite{Chawla:2002} instead.

With this and other, more complex techniques the corresponding authors could enhance the performance gain of their models significantly.
Therefore, future studies should consider more advanced re-sampling strategies and examine their impact on the learning phase of a model.

\subsection{Search Technique}
\label{chp:future_work:sec:hyperparameters:search_technique}
Hyperparameters can be obtained by different techniques.
In the study we use a basic grid-search.
According to \textcite{Bergstra:2012} not all hyperparameters influence am model's performance equally and therefore grid-search is a bad choice.
Recent research has proposed different alternatives for a simple grid-search, for example \textit{random search} \parencite{Bergstra:2012}.
Instead of defining a parameter grid and using the combination with which the best performance was achieved, with random search one randomly picks parameters within an range.

Since the used hyperparameters directly impact a model's performance \parencite{Claesen:2015}, future research should consider the usage of more advanced techniques for the hyperparameter search like random search.
