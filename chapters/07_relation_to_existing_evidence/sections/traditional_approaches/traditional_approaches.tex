\section{Traditional Approaches}
\label{chp:relation_to_existing_evidence:sec:traditional_approaches}

Todo: Mover \parencite{Femmer:2017} here.

% why are results bad
When we compare our results from \cref{chp:study:sec:results} with more \hyperref[chp:related_research:sec:rule_based approaches]{\textit{traditional approaches}}, one immediately observes that our models perform rather poorly.
One example is Smella, developed by \textcite{Femmer:2017}.
It uncovers \textit{requirement smells} with a precision of $59\%$ and a recall of $82\%$ whereas our \ac{BERT} based model achieves same recall, but only a precision of $0.36$.
In contrast to our \textit{vague requirements} Smella's objective is to identify requirement smells.
Although these objectives are not identical, they are very similar.
Therefore, we can follow that Smella as an example and rule based approaches in general are severely better in identifying poor requirements.
