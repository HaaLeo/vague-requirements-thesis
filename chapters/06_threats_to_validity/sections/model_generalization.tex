\section{Model Generalization}
\label{chp:threats_to_validity:sec:model_generalization}

Within the study we calculated the metrics used for evaluation and comparison to other models with the datasets $D_{crowd_{test}}$ and $D_{all_{test}}$.
The crowdsourced dataset consists of $29$ vague and $126$ not-vague requirements.
The overall test dataset is almost twice as big and includes $59$ requirements annotated as vague and $219$ requirements labeled as not-vague.
The properties of the test datasets are visualized in \cref{tab:threats_to_validity:sec:model_generalization:test_datasets}.

\begin{table}[htpb]
    \centering
    \begin{tabular}{l|llll}
        \toprule
        Dataset & Vague & Not-Vague & Overall & Proportion Vague Labels\\
        \midrule
        $D_{crowd_{test}}$ & 29 & 126 & 155 & $\approx19\%$\\
        $D_{all_{test}}$ & 59 & 219 & 278 &  $\approx21\%$\\
        \bottomrule
    \end{tabular}
    \caption[Overview of test datasets]{Overview of the test datasets and their properties.}\label{tab:threats_to_validity:sec:model_generalization:test_datasets}
\end{table}

The test datasets include requirements from various domains \parencite{Kummeth:2020} and are unknown to the models.
Nevertheless, they consist of only $29$ and $59$ vague requirements and it is questionable whether those requirements represent all facets of vagueness.
Therefore, it is unsure whether the results obtained with the test datasets sufficiently represent the models' generalization capabilities.
Since we do not have access to more requirements to further investigate this doubt, this remains a threat to validity.
