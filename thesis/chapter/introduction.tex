\chapter{Introduction}
\label{chp:Introduction}
A software product is only as good as its development process \citep{Hsia:1993}.
This process involves specifying and understanding requirements correctly as an integral part.
Research has shown that this is prone to faults which can cost additional time and money \citep{Mendez:2006} and may lead to severe project delay \citep{Femmer:2014}.
It is therefore desirable to avoid those drawbacks by recognizing misleading requirements at an early stage and, in the best case, by specifying them clearly before the next development step.
To resolve those one must rephrase the requirements in an unambiguous way.
However, this is non trivial since often domain knowledge is required to uncover and resolve the issues \citep{Femmer:2017}.
For example in a requirement containing \textit{vague pronouns} like "The software must include a service \textit{which} must be accessible via a user interface." it is unclear whether \textit{which} relates to \textit{The software} or \textit{service}.
Another example for a requirement defect are \textit{loopholes}.
A requirement stating that the software should be tested \textit{as far as possible} leaves the reader room for interpretation and thus is ambiguous.
The previously presented defects called \textit{vague pronouns} and \textit{loopholes} are examples for so called \textit{Requirement Smells} defined by \citet{Femmer:2017}.
If a requirement smell is fulfilled it indicates that a requirement is of insufficient quality.
Consequently, requirement smells can be used to argue about the a requirement's quality

Before one can reason about a requirement's quality, one must check whether requirement smells apply or not.
An approach to accomplish this are manual review.
According to \citet{Salger:2013}, reviews have several drawbacks.
The review must be carried out by the relevant stakeholders and they must fully understand each requirement.
Consequently, the reviewer needs domain knowledge in order to perform the reviews which makes the review more difficult to execute.
Furthermore, the result of a review depends on the reviewer him/herself \citep{Zelkowitz:1983}.
Moreover, the reviewer can be distracted by the earlier mentioned requirement smells.
Therefore, \citet{Femmer:2017} conclude that reviews are costly and time consuming.

Knowing reviews are costly regarding time and money, tooling which supports the requirements engineering would be beneficial for the quality assurance process.
Such tooling could support the reviewer by automatically indicating requirement smells and therefore speed up reviews.
Further, in reality not only dedicated requirement smells are of interest, but more general speaking ambiguous or vague requirements.
Such an assistant tool should be capable to process natural language since requirements are mostly formulated in natural language \citep{Mich:2004} and indicate whether a requirement is vague or not.

Machine Learning (ML) achieved in recent history remarkable results for natural language processing (NLP) tasks \citep{Khan:2016}.
An example to mention is Google's neural network (NN) model called BERT which showcased how NNs can improve performance in transfer learning tremendously \citep{Devlin:2018}.
This recent success shows that NNs have great potential in transfer learning and consequently in detecting vague requirements.

The aim of this thesis is to further explore the potential of modern NNs in the context of detection of vague requirements.
I want to contribute the following points.
\begin{compactenum}
	\item Item 1
	\item Item 2
	\item Item 3
\end{compactenum}
