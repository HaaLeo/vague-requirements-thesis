\RequirePackage[l2tabu,orthodox]{nag}

\documentclass[headsepline,footsepline,footinclude=false,oneside,fontsize=11pt,paper=a4,listof=totoc,bibliography=totoc]{scrbook} % one-sided

\PassOptionsToPackage{table,svgnames,dvipsnames}{xcolor}
% \setcounter{secnumdepth}{5} % To enable numbering for subsubsections
% \setcounter{tocdepth}{5}
\usepackage[utf8]{inputenc}
\usepackage[T1]{fontenc}
\usepackage[sc]{mathpazo}
\usepackage[ngerman,american]{babel}
\usepackage[autostyle]{csquotes}
\usepackage[%
  backend=biber,
  url=false,
  style=alphabetic,
  maxnames=1,
  % minnames=3,
  maxbibnames=99,
  giveninits,
  uniquename=init]{biblatex}
\usepackage{graphicx}
\usepackage{scrhack} % necessary for listings package
\usepackage{listings}
\usepackage{lstautogobble}
\usepackage{tikz}
\usepackage{pgfplots}
\usepackage{pgfplotstable}
\usepackage{booktabs}
\usepackage[final]{microtype}
\usepackage{caption}
\usepackage[hidelinks]{hyperref} % hidelinks removes colored boxes around references and links
\usepackage{amsmath} % for equation alignment
\usepackage[noabbrev, nameinlink]{cleveref} % For \cref lowercase referencing
\usepackage{longtable} % used by acro
\usepackage{acro} % For abbreviations
% \usepackage{diagbox} % Cells with diagonal lines in tables
\usepackage{makecell} % line break in table cells
\usepackage{bm} % Bold math symbols
% To construct confusion matrix
\usepackage{hhline}
\usepackage{multirow}

\usepackage{enumitem} % To configure lists

\bibliography{bibliography}

\setkomafont{disposition}{\normalfont\bfseries} % use serif font for headings
\linespread{1.05} % adjust line spread for mathpazo font

% Add table of contents to PDF bookmarks
\BeforeTOCHead[toc]{{\cleardoublepage\pdfbookmark[0]{\contentsname}{toc}}}

% Define TUM corporate design colors
% Taken from http://portal.mytum.de/corporatedesign/index_print/vorlagen/index_farben
\definecolor{TUMBlue}{HTML}{0065BD}
\definecolor{TUMSecondaryBlue}{HTML}{005293}
\definecolor{TUMSecondaryBlue2}{HTML}{003359}
\definecolor{TUMBlack}{HTML}{000000}
\definecolor{TUMWhite}{HTML}{FFFFFF}
\definecolor{TUMDarkGray}{HTML}{333333}
\definecolor{TUMGray}{HTML}{808080}
\definecolor{TUMLightGray}{HTML}{CCCCC6}
\definecolor{TUMAccentGray}{HTML}{DAD7CB}
\definecolor{TUMAccentOrange}{HTML}{E37222}
\definecolor{TUMAccentGreen}{HTML}{A2AD00}
\definecolor{TUMAccentLightBlue}{HTML}{98C6EA}
\definecolor{TUMAccentBlue}{HTML}{64A0C8}

% Settings for pgfplots
\pgfplotsset{compat=newest}
\pgfplotsset{
  % For available color names, see http://www.latextemplates.com/svgnames-colors
  cycle list={TUMBlue\\TUMAccentOrange\\TUMAccentGreen\\TUMSecondaryBlue2\\TUMDarkGray\\},
}

% Settings for lstlistings
\lstset{%
  basicstyle=\ttfamily,
  columns=fullflexible,
  autogobble,
  keywordstyle=\bfseries\color{TUMBlue},
  stringstyle=\color{TUMAccentGreen}
}

% Set longtable indentation
\setlength\LTleft{10pt}

% Avoid hyphenation (splitting of "natural language processing" for instance)
\tolerance=1
\emergencystretch=\maxdimen
\hyphenpenalty=10000
\hbadness=10000

% acro
\acsetup{
  list/display = used,
  make-links = true,
  format/first-long = \textit,
  list/template = longtable,
}

% ======================================================================
% Declare acronyms
% ======================================================================
\DeclareAcronym{NN}{short = NN , long = neural~network}

\DeclareAcronym{NLP}{short = NLP , long = natural~language~processing}

\DeclareAcronym{ML}{short = ML , long = machine~learning}
% ======================================================================


\newcommand*{\getUniversity}{Technical University of Munich}
\newcommand*{\getFaculty}{Department of Informatics}
\newcommand*{\getTitle}{Detecting Vague Requirements with Machine Learning}
\newcommand*{\getTitleGer}{Detektion von vagen Anforderungen mit maschinellem Lernen}
\newcommand*{\getAuthor}{Leo Hanisch}
\newcommand*{\getDoctype}{Master's Thesis in Robotics, Cognition, Intelligence}
\newcommand*{\getSupervisor}{Prof. Dr. Dr. h.c. Manfred Broy}
\newcommand*{\getAdvisor}{Dr. rer. nat. Sebastian Eder}
\newcommand*{\getSubmissionDate}{November 15, 2020}
\newcommand*{\getSubmissionLocation}{Munich}

\begin{document}

% Set page numbering to avoid "destination with the same identifier has been already used" warning for cover page.
% (see https://en.wikibooks.org/wiki/LaTeX/Hyperlinks#Problems_with_Links_and_Pages).
\pagenumbering{alph}
\input{pages/cover}

\frontmatter{}

\input{pages/title}
\input{pages/disclaimer}
\input{pages/acknowledgments}
\chapter{\abstractname}

Requirement engineering is an integral part of the modern software engineering process.
Within this process many mistakes can occur which cost a lot of time and money in the subsequent development steps.
Therefore, one should strive to identify misleading requirements as soon as possible.

Traditional approaches try to identify misleading requirements based upon different rule sets.
However, machine learning achieved remarkable results on different natural language processing tasks.
Therefore, we want to explore whether and to what extent state of the art machine learning approaches are capable of uncovering problematic requirements.

With this thesis we contribute a dataset of requirements which are labeled as vague or not-vague.
Further, we evaluate the ability of transformer-based machine learning models to classify requirements.
These models are vanilla implementations of different transformers and should be considered as baseline for future research.

Our models achieve an $F_1$ score of $0.5$ which is worse than the performance of other rule-based approaches
We conclude that further steps must be taken until transformer\nobreakdash-based models can compete with traditional approaches.

\microtypesetup{protrusion=false}
\tableofcontents{}
\microtypesetup{protrusion=true}

\mainmatter{}
% ## Struktur
%  1. Einleitung
%     1.1 Motivation:
%     1.2 Problem:
%     1.3 Mein Beitrag: Implementierung und ob Algo wirkt
%     1.4 Warum löst das das Problem? Weil Studie systematisch durchgeführt 1 Seite
%   1. Abstract
%     Jeder der Punkte auf 1 bis zwei Sätze zusammengefasst
%  1. Fundamentals/Grundlagen: Erklärungen, Definitionen 2/3 Seiten
%    1.1 Was ist bug prediction
%    1.2 was ist ein linear classifier
%    1.3 Ansatz Algo für Bug-prediction erklären
%  1. Related Work (Pro Quelle ein Absatz) 7 bis 10 Papiere
%      1. Studien über bug prediction nur statisch hard gecodede Muster (google scholar)
%  (1. Approach) wie ich classifier traniere
%  1. Studie (eigentlicher Beitrag der ARbeit)
%      1.1 Ziel der Studie (Forschungsfragen)
%          Forschungsfragen(Funktioniert der Algorithmus)
%      1.1 Studie design (Vorgehen)
%          1.1 git auschecken
%          1.2 Algo drauf loslassen
%          1.3 Selbst korrigieren
%          1.4 Präzision bestimmen ("Frage 1 wird beantwortet mit folgender Metrik(Formel)")
%  1. Studie Objekts (besser mehrere)
%      1.1 Welche Repos habe ich verwendet
%      1.3 Wieviele Commits welche programmiersprache wieviele devs wieviele lines of code
%      Tabelle mit Charakteristika
%  (1. Studie exekution)
%  1. Studie Results (gegliedert nach Forschungsfrage)
%     1.1 nur die Zahlen keine Interpretation (recht kurz pro Studie Objekt eine Tabelle oder so)
%  1. Interpretation der Ergebnisse (Zeigen wie gescheit ich bin :D)
%     Antworten auf Forschungsfragen
%  1. Threats to Validity (Angriffspunkte auf meine Studie)
%      1. Veralgemeinerbarkeit weil nur begrenzte studie objekts oder nur eine Programmiersprache
%      1. Könnte Fehler in Algorithums
%      1. Ich könnte biased sein
%      (externe Validität interne Validität)
%      Ist kleiner Threat weil...
%  1. Schluss (Zusammenfassung von Einleitung bis Schluss)

%  Studie zwischen 5 bis 8 Seiten
%  Studie design ein Bild
%  zwischen 10 und 20 Seiten
% Ein Gedanke pro Absatz nicht mehr als 8-10 Zeilen

% Vortrag
% Einleitung mit Motivation
% Ansatz vorstellen
% Studie vorstellen
% Studie design
% Studie result (1. 5 bis 10 dateien sind studie objekts)
% Präzi schwierig
% Aber aufgrund der "natur der Datei"

% TODO: Remove me
% !TeX root = ../main.tex
% Add the above to each chapter to make compiling the PDF easier in some editors.

\chapter{Introduction}
\label{chp:introduction}

Recent research found that a software product is only as good as its development process \parencite{Hsia:1993}.
Correctly specifying and understanding requirements is an integral part of this process, known as \ac{RE}.
Research has shown that \ac{RE} is prone to faults which can cost additional time and money \parencite{Mendez:2016} and may lead to severe project delay \parencite{Femmer:2014}.
Further, the later design changes are introduced in the development process, the costlier these changes are \parencite{Folkestad:2001}.
\Cref{fig:introduction:relation} visualizes the relation of the costs of a change to the development process phase.
\begin{figure}[htpb]
    \centering
    \def\svgwidth{\columnwidth}
    \input{figures/introduction/Relation.pdf_tex}
    \caption[Relation of Development Phase to Cost per Change]{The relation of the development phase to the cost of design changes adapted from \textcite{Folkestad:2001}}\label{fig:introduction:relation}
\end{figure}

It is therefore desirable to avoid those drawbacks by recognizing misleading requirements at an early stage so that faulty requirements can be reformulated in an unambiguous way before the next development step.
However, this is not trivial since often domain knowledge is required to uncover and resolve the issues \parencite{Femmer:2017}.
For example in "The software must include a service \textit{which} must be accessible via a user interface." it is unclear whether \textit{which} relates to \textit{The software} or \textit{service}.
In this case the requirement contains a requirement defect called \textit{vague pronouns}.
Another example for a requirement defect is that of \textit{loopholes}.
A requirement stating that the software should be tested \textit{as far as possible} leaves the reader room for interpretation and thus is ambiguous.
The previously presented defects called \textit{vague pronouns} and \textit{loopholes} are examples for so called \textit{Requirement Smells} defined by \textcite{Femmer:2017}.
Requirement smells can be used to assess a requirement's quality.
If a requirement smell is fulfilled it indicates that a requirement is of insufficient quality.

A common approach to check whether a requirement smell applies or not are manual reviews.
According to \textcite{Salger:2013}, reviews have several drawbacks.
The review must be carried out by the relevant stakeholders and they must fully understand each requirement.
Consequently, the reviewer needs domain knowledge in order to perform the reviews which makes the review more challenging.
Furthermore, the result of a review depends on the reviewer him/herself \parencite{Zelkowitz:1983} and the reviewer can be distracted by the earlier mentioned requirement smells themselves.
Therefore, \textcite{Femmer:2017} conclude that manual reviews are costly and time consuming.

Tooling which supports the review process could consequently have the potential to save resources and further benefit the quality assurance process.
Such tooling could for example support the reviewer by automatically indicating requirement smells and therefore speed up reviews.
Further, in industrial environments not only dedicated requirement smells are of interest but, more generally speaking, ambiguous or vague requirements.
Because requirements are mostly formulated in natural language \parencite{Mich:2004} an assisting tool has to be capable of processing natural language and then assess whether a requirement is vague or not.

In recent history \Ac{ML} achieved remarkable results for complex \ac{NLP} tasks \parencite{Khan:2016}.
An important example to mention is Google's \ac{NN} model called BERT which showcased how \acp{NN} can improve performance in \textit{transfer learning} tremendously \parencite{Devlin:2018}.
This recent success indicates that \acp{NN} have great potential in transfer learning and therefore could be capable to detect vague requirements.
However, recent research lacks the exploration of vague sentences and in particular vague requirements.

The aim of this thesis is to further explore the capabilities of modern \acp{NN} in the context of detection of vague requirements and whether they are suitable candidates to improve the quality assurance process.
In order to improve \ac{RE} we make the following contributions:
\begin{enumerate}
    \item Since recent research endorses \acp{NN}' capabilities to solve complex tasks but has not yet explored this in the context of vague requirements, this thesis evaluates whether and to what extent state of the art \acp{NN} are capable to classify requirements as \textit{vague} and \textit{not vague}.
    \item To successfully apply transfer learning one needs a set of labeled datapoints. However, there are no such datasets publicly available.
        Therefore, we create a dataset containing requirements and labels which indicate whether the corresponding requirement is vague or not.
    \item It is known that recent \ac{ML} approaches perform well on \ac{NLP} tasks.
        However, they have not been compared among each other in the context of vague requirements.
        Therefore, we compare different \ac{ML} approaches among each other to determine which performs best using dedicated metrics.
\end{enumerate}

The thesis is structured as follows.
In \cref{chp:fundamentals} we establish the knowledge fundamentals which are required throughout this thesis.
After that, we examine related research in the field of \ac{RE}.
The specific approach we use is introduced in \cref{chp:approach} and its sections.
To assess how good this approach is we carry out a study which we describe in depth in \cref{chp:study}.
Threats to the study's validity are presented in \cref{chp:threats_to_validity}.
\Cref{chp:relation_to_existing_evidence} includes the comparison of our approach and its results to other approaches.
In \cref{chp:future_work} we derive opportunities for future research.
The conclusion of this thesis is given in \cref{chp:conclusion}.

\chapter{Fundamentals}
\label{chp:fundamentals}

This chapter aims to establish the fundamental knowledge which is required to understand this thesis.
First we will introduce vagueness in the context of \ac{RE} to specify what we consider a vague requirement.
After that, we present a method named crowdsourcing to rapidly generate a labeled dataset.
To evaluate whether transformers are capable to classify vague requirements we use dedicated metrics which are explained in \cref{chp:fundamentals:sec:metrics}.
Since our chosen approach bases on state of the art \ac{ML} concepts and \acp{DNN} we explain those in \cref{chp:fundamentals:sec:machine_learning}.
As last part of the fundamentals we introduce \ac{IRA} in \cref{chp:fundamentals:sec:inter_rater_agreement} in order to judge the quality of an annotated dataset.

\section{Vagueness}
\label{chp:fundamentals:sec:vagueness}
In general, vagueness describes some kind of uncertainty.
However, it is quite difficult to give a precise and concise definition of vagueness.
Therefore, this chapter gives an overview how vagueness is defined across other research fields.
At the end of this chapter it is defined what a \textit{vague requirement} is.

The first discipline to look at is philosophy.
According to \textcite{Braun:2007} an expression is vague when a speaker cannot determine whether the expression applies correctly although he/her knows all relevant information.
An example is a series of different color panels which continuously vary from blue to yellow.
Although one can clearly see each patch, it will be hard to distinguish whether some intermediate color panel is blue or yellow.
\textcite{Braun:2007} \textit{"think that vagueness occurs when there exist multiple equally good candidates to be the meaning of a given linguistic expression"}.
Applying their understanding of vagueness in the upper example "Blue" is vague since their exist multiple colors which are referred to as "Blue".

\section{Crowdsourcing}
\label{chp:fundamentals:sec:crowdsourcing}
Manually labeling large datasets is a tedious task and could take weeks or even months \parencite{Welinder:2010}.
According to \textcite{Welinder:2010}, this could also include the training of people on custom interfaces and one must ensure that the annotators stay motivated in order to produce high quality annotations.
One way to overcome this limitations ist the usage of \textit{crowdsourcing}.
With crowdsourcing dozens of people can access the annotation task and contribute to the dataset.
\textcite{Howe:2008} defines crowdsourcing as \textit{"the act of taking a task traditionally performed by a designated agent (such as an employee or a contractor) and outsourcing it by making an open call to an undefined but large group of people"}.
Examples for crowdsourcing tasks are the annotation of images or the execution of surveys.
Throughout this work we follow the definition of \textcite{Howe:2008} for crowdsourcing.

\section{Metrics}
\label{chp:fundamentals:sec:metrics}

In order to evaluate classification models usually different metrics are used.
This chapter introduces the metrics which we use to evaluate the trained models.

\begin{figure}[htpb]
    \centering
    \def\svgwidth{\columnwidth}
    \input{figures/metrics/True_Positives.pdf_tex}
    \caption[Visualization of True Positives]{A visualization of true positives, true negatives, false positives and false negatives.}\label{fig:metrics:tp_vis}
\end{figure}


\subsection{Precision}
\label{chp:fundamentals:sec:metrics:subsec:precision}

\subsection{Recall}
\label{chp:fundamentals:sec:metrics:subsec:Recall}

\subsection{Average precision}
\label{chp:fundamentals:sec:metrics:subsec:average_precision}


\section{Machine Learning}
\label{chp:fundamentals:sec:machine_learning}
In this section we want to establish the fundamentals regarding \ac{ML} needed throughout this thesis.
We presume that the reader has basic knowledge on the domain of \ac{ML} and \acp{NN}.
If one wishes to brush up his/her knowledge on this domains we recommend \citeauthor{Alpaydin:2020}'s Introduction to Machine Learning \parencite{Alpaydin:2020}.
Instead, here we want to close the gap between basic \ac{ML} concepts and state of the art techniques.


\subsection{Transformer}
\label{chp:fundamentals:sec:machine_learning:subsec:transformer}
The \textit{transformer} was introduced in \citeyear{Vaswani:2017} by \textcite{Vaswani:2017}.
In this subsection we want to introduce the transformer and its architecture.

\subsubsection{High Level Architecture}
As its name suggests a transformer takes an input sequence and transforms it to a different output sequence.
In the context of machine translation it could take the sequence "Tom went home, because he was tired." and translate it to another language, for example to german: "Tom ging nach Hause, weil er müde war.".
The transformer consists of two main components: A stack of \textit{encoders} and a corresponding \textit{decoder} stack.
\Textcite{Vaswani:2017} use six encoders and six decoders in each stack, but the value six is arbitrary and one can use more or less en-/decoders.
\Cref{fig:fundamentals:machine_learning:transformer} visualizes the high level architecture of a transformer which consists of arbitrary many encoders and decoders.
Further, it shows an example input and the corresponding output for a machine translation task.
\begin{figure}[htpb]
    \centering
    % \def\svgwidth{\columnwidth} % Scale for height instead
    \input{figures/machine_learning/Transformer.pdf_tex}
    \caption[High Level Transformer Architecture]{The high level architecture of a transformer.}\label{fig:fundamentals:machine_learning:transformer}
\end{figure}

\subsubsection{Encoder}
In the following, we focus on the encoders, because we do not further use decoders in this work.
Before the input can be passed to the encoder stack it must be embedded, meaning that the input sequence is converted to a tensor.
This tensor is then passed through the encoder stack.
One encoder itself consists of an attention layer and and a subsequent fully connected feed forward \ac{NN} as shown in fig \cref{fig:fundamentals:machine_learning:encoder}.
% \pagebreak % Todo check page breaks
\begin{figure}[htpb]
    \centering
    % \def\svgwidth{\columnwidth} % Scale for height instead
    \input{figures/machine_learning/Encoder.pdf_tex}
    \caption[High Level Encoder Architecture]{The high level architecture of a single encoder.}\label{fig:fundamentals:machine_learning:encoder}
\end{figure}

\subsubsection{Self Attention}
The layer which enables the transformer to perform very well on a wide range of \ac{NLP} tasks is its attention layer.
\Textcite{Vaswani:2017} describe attention \textit{"as mapping a query and a set of key-value pairs to an output, where the query, keys, values, and output are all vectors"}.
The authors use the \textit{Scaled Dot-Product Attention}.
It is built by computing the dot product of all keys with all queries.
Then it is scaled by $\frac{1}{\sqrt{d_k}}$ where $d_k$ is the dimension of the keys.
According to the authors, this leads to more stable gradients.
The values' weights are obtained by applying the softmax function.
Given the query matrix $\bm{Q}$, key matrix $\bm{K}$ and a matrix $\bm{V}$ to represent the values, attention is computed by \cref{eq:attention}.


\begin{equation}\label{eq:attention}
    \text{Attention}(\bm{Q},\bm{K},\bm{V}) = \text{softmax}(\frac{\bm{Q}\bm{K}^T}{\sqrt{d_k}}) \bm{V}
\end{equation}

\subsubsection{Multi Head Attention}
According to \textcite{Vaswani:2017} it is beneficial to project all values, keys and queries $h$ times instead of computing a single attention function with $d_{model}$ dimensional values, keys, and queries with $d_{model}$ being the model's output dimension.
They perform one attention function on each of the projected versions and concatenate their outputs.
These outputs are then once more projected to obtain the final values.
The calculation of multi head attention is shown in \cref{eq:multi_head_attention}.
\begin{equation}\label{eq:multi_head_attention}
    \begin{aligned}
        \text{MultiHead}(\bm{Q},\bm{K},\bm{V}) &= \text{Concat}(head_1,\dots, head_h)\bm{W}^O\\
        \text{where }head_i &= \text{Attention}(\bm{Q}\bm{W}^Q_i, \bm{K}\bm{W}^K_i, \bm{V}\bm{W^V_i})
    \end{aligned}
\end{equation}

With $d_v$ being the dimension of the values and the parameter matrices $\bm{W}^K_i \in \mathbb{R}^{d_{model} \times d_k}$, $\bm{W}^Q_i \in \mathbb{R}^{d_{model} \times d_k}$, $\bm{W}^V_i \in \mathbb{R}^{d_{model} \times d_v}$ and $\bm{W}^O_i \in \mathbb{R}^{hd_v \times d_{model}}$.
This mechanism \textit{"allows the model to jointly attend to information from different representation subspaces at different positions"} \parencite{Vaswani:2017}.
In the example of the input sequence "Tom went home, because he was tired.", when encoding "he" multi head attention allows the model to attend to "Tom" and incorporate this information.

\subsection{Transfer Learning}
\label{chp:fundamentals:sec:machine_learning:subsec:transfer_learning}
Transfer Learning

\subsection{Local Interpretable Model-Agnostic Explanations}
\label{chp:fundamentals:sec:machine_learning:subsec:transfer_learning}
According to \textcite{Ribeiro:2016} \ac{ML} models are widely spread, although they mostly remain black boxes.
Further, they state that is essential to be able to trust a model when it used to make decisions.
However, it is very difficult for a human to trust a model which cannot explain its predictions.
To address this issue \textcite{Ribeiro:2016} introduce \ac{LIME}.

When one wants to apply \ac{LIME} for a sample $s$ to obtain an explanation for the prediction it first samples datapoints near $s$ and weights them with their distance to $s$.
Then the predictions for the sampled datapoints are generated by the original model.
This labeled dataset is used to train an interpretable linear model which approximates the original well near $s$.
Consequently, \ac{LIME} presumes that a complex model is linear on a local scale.
The explanations for the prediction of $s$ of the original model are then derived using the interpretable model. \parencite{Ribeiro:2016}


\section{Inter Rater Agreement}
\label{chp:fundamentals:sec:inter_rater_agreement}
Many \ac{ML} models require a dataset which is used to train the models.
This data is often generated by multiple rater which assign a label to each data point.
The used dataset directly influences the \ac{ML} model \parencite{Gray:2011} and therefore, well designed research studies must include mechanisms to capture \ac{IRA} \parencite{McHugh:2012}.

This chapter gives an introduction to different \ac{IRA} metrics and their applications.

\subsection{Cohen's Kappa}
\label{chp:fundamentals:sec:inter_rater_agreement:subsec:cohens_kappa}
Cohen's Kappa $\kappa$ was introduced by \textcite{Cohen:1960} in 1960.
He states the hypothesis that even if all raters are unaware of the correct answer and purely guessing, nevertheless some data points are congruent.
In his opinion that random congruency should be considered by agreement statistics.
To tackle this issue he introduced the kappa statistics to account for the random agreement among raters.
Similar to other correlation statistics can take values in the range from -1 to 1.
0 indicates the agreement obtained by random choice, whereas 1 represents perfect agreement.
The kappa calculation includes two quantities.
$p_o$ is the observed agreement of raters and $p_e$ is the probability of chance agreement.
The overall formula of Cohen's $\kappa$ is then given by \cref{eq:Cohens_kappa}:
\begin{equation}\label{eq:Cohens_kappa}
    \kappa = \frac{p_o - p_e}{1 - p_e}
\end{equation}
% TODO link to accuracy definition
$p_o$ is given by the experiments \textit{accuracy}.
For $p_e$
% next step get formula for p_e

\subsection[Scott's Pi]{Scott's $\pi$}
\label{chp:fundamentals:sec:inter_rater_agreement:subsec:scotts_pi}
\textcite{Scott:1955} developed another inter-observer agreement metric.
It was introduced specifically to measure the agreement for survey research.
This field of research includes annotating textual entities with classes by different annotators which is a common use case in \ac{NLP}.
Its aim is to measure the agreement among raters' multiple responses which are classified in exclusive categories.
According to \textcite{Scott:1955}, the percent of answers which the raters agree on as well as the \textit{consistency index S} introduced by \textcite{Bennett:1954} are biased.
Under the assumption that annotators have the same distribution of answers, he introduced his index $\pi$, referred to as \textit{Scott's $\pi$}, which has the same formula as \hyperref[chp:fundamentals:sec:inter_rater_agreement:subsec:cohens_kappa]{Cohen's kappa} (\cref{eq:Cohens_kappa}).

\begin{equation}\label{eq:Scotts_pi}
    \pi = \frac{P_o - P_e}{1 - P_e}
\end{equation}

Here $P_o$ is again the observed percentage of agreement and $P_e$ is the percentage of agreement which can be expected merely by chance.
Similar to Cohen's kappa Scott's $\pi$ is limited to two raters.
The only difference of Scott's $\pi$ to Cohen's kappa is the calculation of the by chance expected agreement $P_e$.
In contrast to \textcite{Cohen:1960} who uses the squared geometric mean of marginal proportions, \textcite{Scott:1955} used their squared arithmetic mean.
Following the notation of \cref{tab:cohens_kappa_sample_definition} this yields:

\begin{equation}\label{eq:Scotts_pi:p_e}
    P_e = \sum_{i=1}^{C} (\frac{p_{A, c_i} + p_{B, c_i}}{2})^2
\end{equation}

Applying Scott's $\pi$ to the sample data listed in \cref{tab:cohens_kappa_sample_data} one obtains the same $P_o=0.7$ but a different $P_e = (\frac{0.5 + 0.6}{2})^2 + (\frac{0.4+0.5}{2})^2 = 0.505$.
Using $P_o$ and $P_e$, yields Scott's $\pi = \frac{0.7 - 0.505}{1-0.505} = 0.\overline{39}$.

\subsection{Free-Marginal Multirater Kappa}
\label{chp:fundamentals:sec:inter_rater_agreement:subsec:free_marginal_multirater_kappa}
This is Randolph



\chapter{Related Research}
\label{chp:related_research}
Identifying defects in requirements is an active and ongoing research field.
This chapter presents different approaches that tackle this challenge.
The approaches are categorized regarding the mechanism they utilize to solve the task.
It is distinguished among \textit{rule-based} and \textit{\ac{ML} based} approaches.

\section{Rule-Based Approaches}
\label{chp:related_research:sec:rule_based approaches}

\textcite{Gleich:2010} present a tool to enable automated ambiguity detection.
Moreover, the tool shall help its users to create awareness that the detected problems are earnest and explain the cause of the detected ambiguous phrases to train the users.
The tool is similar to Unix' \textit{grep} command line tool.
%TODO print grep in "coding font"
It analyzes each line and applies recognizes patterns using regular expressions.
Each matched ambiguity is categorized according to its severity.
All matches are highlighted and the users get additional information when they hover over a match explaining the found ambiguity.
The tool was evaluated using 50 German as well as 50 English sentences and achieved a precision up to 97\% and 86\% recall.

\section{Machine Learning Based Approaches}
\label{chp:related_research:sec:machine_learingn_based_approaches}

Apart from solely rule-based approaches, \ac{ML}-based approaches exist to classify requirements or more generally, documents.
In this chapter \ac{ML}-based approaches for document classification are presented.

\subsection{Hedge Classification in Scientific Literature}
\textcite{Medlock:2007} present an approach which examines hedge classification in scientific literature.
The subject of hedge classification is to detect speculative language.
First, they introduce a probabilistic model to gather training data.
The same model is used to derive a weakly supervised learner which classifies a sample as speculative when its probability exceeds a predefined threshold.
This iterative learning approach is capable of performing hedge classification with similar accuracy as other approaches which all use a \ac{SVM}.
They conclude that their simple \ac{ML} based approach is capable of performing hedge classification and that more complex \ac{ML} approaches are likely to perform even better.

\subsection{Decision Trees}
\textcite{Ormandjieva:2007} present an approach which utilizes decision trees for the classification task.
Their quality model distinguishes between \textit{surface understanding} and \textit{conceptual understanding}.
Surface understanding captures the literal meaning like how difficult or easy it is to understand a requirements documents' passages, whereas conceptual understanding aims for a passage's interpretation.
For example how difficult it would be for a developer to implement a system by only reading the paragraph.

The dataset was obtained by manual reviews of four annotators.
Each annotator carefully read a set of requirements documents and classified their passages once with respect to surface understanding and once with respect to conceptual understanding.
The \ac{IRA} is indicated by Cohen's Kappa \parencite{Cohen:1960} and is 0.64 for conceptual understanding and 0.66 for surface understanding.

For the classification task each sentence of the requirements documents was \ac{POS} tagged and its syntax parsed using the Stanford Parser \parencite{Klein:2002}.
Then the number of occurrences of the indicators for little surface understanding were counted.
Based on that information a decision tree was constructed.
The authors mention that a \ac{NN} could maybe achieve better results.
However, due to the lack of data they prefer a decision tree.
The trained decision tree was able to solve the classification task with 86.67\% accuracy.
\textcite{Ormandjieva:2007} conclude that it is indeed feasible to uncover faults related to surface understanding using a decision tree and see their quality model approved as suitable.

\subsection{Conditional Random Fields}
\textcite{Yang:2012} use \ac{CRF} to extract uncertainty cues from requirements documents.
They manually labeled several requirements documents for uncertainty cues and their corresponding scopes.
The detection problem was formulated as a labelling task of a sequence on token-level.
Each word of a sentence is assigned a class label indicating whether the word is the first word of the cue, inside a cue or not in a cue.
To additionally gather further information regarding the semantics of a cue keyword, more linguistic features were extracted.
Examples are the word lemma, \ac{POS} tag, \ac{POS} tags of the three neighboring words and grammatical relations.

Those features and the class label are used to train a classification model using the \ac{CRF} algorithm \parencite{Lafferty:2001}.
Further, a post-processing step is applied to capture infrequent cues extracted from the training dataset.
In this step the cues are detected using string matching.
When a match occurs the sentence is classified as speculative.
In the conducted study multiple models with a different amount of features were trained and tested.

Speculative sentences were detected with 85.58\% precision and 77.65\% recall considering all features.
Regarding uncertainty scope identification the presented system performs worse: 54.37\% precision and 49.95\% recall were achieved.
They conclude that their implemented approach works well on uncovering speculative sentences and suggest that different supervised \ac{ML} techniques could be promising in solving this task.

\subsection{Rule Induction}
\textcite{Parra:2015} use \ac{ML} techniques to classify requirements regarding their quality.
The approach of choice is rule induction.
The authors use this technique because of its robustness against noise due to insufficient data.
To generate the models they use the PART rule induction algorithm \parencite{Eibe:1998}.
The obtained models use a requirement's different metrics to classify it.
The models are used to induce rules which then can be used to evaluate a requirement's quality according to the corresponding metric.

The models were examined using a dataset annotated by experts and achieved an accuracy in a range between 83.27\% and 87.72\%.
The achieved recall remains unknown.
The authors conclude that it is possible to learn the quality features of a requirement and use the obtained models for quality prediction of new requirements.
They endorse enhancing the model creation process to further improve the performance.

\subsection{Linguistic Hedging in the Monetary Political Domain}
\textcite{Stajner:2017} examine linguistic hedging in the monetary political domain.
They create two different datasets.
One includes transcribed \textit{debates} of different meetings of several committees.
The second dataset includes the meetings' statement reports which are typically conducted at the end of a meeting and further referred to as \textit{decisions}.
Three annotators labeled the datasets' entries as \textit{speculative} or \textit{non-speculative}.
The averaged pairwise Cohen's Kappa \parencite{Cohen:1960} was 0.56 for the debates and 0.61 for the decisions dataset.
The annotators marked phrases which indicate a speculative sentence in their opinion.

For the classification task a \ac{SVM} was trained on both datasets.
They used \ac{BoW}, the lists of speculation triggers or different combinations as input features for the \acp{SVM}.
The authors benchmarked their \ac{SVM} approach with a \ac{CNN} and with the best performing system for the CoNLL-2010 shared task \parencite{Farkas:2010}.
They conclude that a general domain dataset can be used to train a model for a hedge classification task in this domain.
Further they showed that their \ac{SVM} approach including lists of speculation triggers performs as well as other state of the art systems in other domains and could outperform the \ac{CNN} classifier on the debates dataset.

\subsection{Summary}
All of the presented works use \ac{ML} techniques in order to improve their results.
Some try to build classifiers which are capable of predicting a new requirement's quality.
Others use those techniques to extract specific cues and their location.
However, none of the mentioned approaches examines how state of the art \acp{NN}, like \acp{DNN}, perform at classifying requirements as vague or not vague.
With this thesis we want to answer the question whether \acp{DNN} are promising in classifying requirements regarding their vagueness.


\chapter{Approach}
\label{chp:approach}
% (wie werde ich die Vagheit detektieren → NN, Transformer Architektur) welche
% Grob vorstellen was ich machen werde:

% ich benutze mehrere NNs um Vagheit zu klassifizieren
% Eingaben, Ausgaben
% eher top Level

In this chapter the concrete approach is presented which was used to detect vague requirements.
To classify requirements as vague or not \acp{NN} are used.
The result of each \ac{NN} will be evaluated using dedicated metrics.

Transformer based approaches have turned out to be viable alternatives to traditional architectures based on \acp{RNN} for \ac{NLP} tasks like language translation \parencites{Gehring:2017}{Vaswani:2017}.
This showcases that transformers are capable to solve complex \ac{NLP} tasks.
Therefore, we also leverage a transformer based approach to attempt to classify vague requirements.
In detail, our approach considers three different models.
Namely \ac{BERT} \parencite{Devlin:2018}, \ac{DistilBERT} \parencite{Sanh:2019} and \ac{ERNIE2.0} \parencite{Sun:2019a}.
The following sub chapters outline which particular models and metrics are used.

\section{BERT}
\label{chp:approach:sec:models:subsec:BERT}

% Paragraph on architecture
\Ac{BERT} is based on \textit{encoder} of the original transformer model of \textcite{Vaswani:2017}.
It is simply a stack of encoders.
There exist two variants of \ac{BERT}: \ac{BERT}$_{base}$ and \ac{BERT}$_{large}$.
The first consists of 12 encoder layers, second of 24 which yields to 110 million parameters and 310 million respectively.
Since the 24 layer variant takes severely longer to train we use the 12 encoder layer variant in this approach. \parencite{Devlin:2018}

% Paragraph on in- / outputs
\Ac{BERT} is suitable for different kinds of downstream tasks, since it is capable to embed also sentence pairs in a single sequence.
This is achieved by introducing special input tokens which have a designated meaning.
For our approach most important special token is the \textit{[CLS]} token, because its corresponding final hidden state represents the aggregation of the input sequence and can be used for further downstream classification tasks.
\Ac{BERT}'s input is the summation of three different embeddings.
The first embedding is the \textit{token embedding}.
Here each input token (for example each word of a sentence) is embedded using WordPiece embeddings \parencite{Wu:2016} which utilize a vocabulary of 30,000 words.
The second embedding is called \textit{segment embedding} and indicates to which input sequence a token belongs.
The third embedding embeds a tokens position.
The input of \ac{BERT} is obtained by summing up those three embeddings for each input token.
% Todo Figure for visualization of embedding creation here?

% Paragraph on downstream classification
Once we embed our input sentences and pass them to \ac{BERT}, \ac{BERT} aggregates the sequence in the output of the [CLS] token.
However, this raw token does not yet indicate whether a requirement is vague or not.
In order to finally classify a requirement, the [CLS] token is passed to a single layer feed forward \ac{NN}.
The output of this layer then indicates in percent how certain the model is whether a requirement is vague or not.
It consists of 1,500 parameters.

\section{DistilBERT}
\label{chp:approach:sec:distilbert}

Parameter-heavy deep learning models like \ac{BERT} tend to raise scaling issues the more parameters they include \parencite{Schwartz:2019}.
To tackle this concern, \textcite{Sanh:2019} introduced \ac{DistilBERT}.

% Paragraph architecture and learning
The architecture of \ac{DistilBERT} is similar to that of \ac{BERT}.
However, \ac{DistilBERT} uses only half as many layers.
To obtain a much smaller model the authors use \textit{knowledge distillation} as proposed by \textcites{Bucilua:2006}{Hinton:2015}.
When applying this technique a smaller student model is trained to replicate the behavior of the larger model, the teacher.
The teacher can also be a set of different models.
With this technique the authors successfully trained \ac{DistilBERT} to keep 97\% of \ac{BERT}'s performance.
Compared to \ac{BERT}, \ac{DistilBERT} consists of 40\% less parameters and trains 60\% faster. \parencite{Sanh:2019}

Since \ac{DistilBERT} and \ac{BERT} share the overall same architecture, \ac{DistilBERT} also exposes a [CLS] token which again can be seen as an aggregation of the input sequence.
Similar to our \ac{BERT}-based model, we use this token as input for a fully connected feed forward \ac{NN} with 1,538 trainable parameters.

\section{ERNIE 2.0}
\label{chp:approach:sec:models:subsec:ernie2.0}

According to \textcite{Sun:2019a} the aim of most pre-training routines are based on the co-occurrence of sentences or words.
Further, they argue that with such a pre-training procedure other valuable information like lexical or semantic information is not considered.
To solve this issue, they proposed a new learning framework called \ac{ERNIE2.0}.
This framework enables a model to learn multiple tasks without forgetting the the knowledge learned by the previous tasks.
To accomplish this, their continual multi-task learning method integrates the new task by initializing a model with the previously learned parameters and then trains for the newly added task together with the prior tasks.

To showcase that their approach works they pre-train a model, called \ac{ERNIE2.0} model, using the \ac{ERNIE2.0} framework.
Like \ac{BERT} and \ac{DistilBERT} the model is based on the transformer architecture of \textcite{Vaswani:2017} and consists of multiple encoder layer.
To ease the comparison with \ac{BERT} they decide to use the same architecture but pre-train it with the \ac{ERNIE2.0} framework.
However, the input of the \ac{ERNIE2.0} model is slightly different.
In addition to the position, segment and token embedding they use another \textit{task embedding}.
The purpose of this new embedding is \textit{"to represent the characteristic of different tasks"} \parencite{Sun:2019a}.
To capture lexical and syntactic information during the training phase they define word-aware and semantic aware pre-training tasks.
The exact explanation of those tasks is not in the scope of this thesis. \parencite{Sun:2019a}

For our classification task we will use the \ac{ERNIE2.0} model which was pre-trained using the \ac{ERNIE2.0} framework.
Due to the similar architecture to \ac{BERT} the \ac{ERNIE2.0} model again has a [CLS] output token which can be used for downstream classification tasks.
Analogical to the prior two models we will use a single layer feed forward \ac{NN} with 1,500 trainable parameters to classify requirements as vague or not.


\chapter{Study}
\label{chp:study}

\section{Goal}
\label{chp:study:sec:goal}
The goal of this study is to explore whether and to what extent state of the art \ac{ML} approaches are capable to detect vague requirements.
In order to reach this goal we want to answer the following \ac{RQ}:

\begin{description}
    \item[\Ac{RQ}:] Are transformer based \ac{ML} models in combination with transfer learning suitable to correctly classify requirements as vague or not-vague?
\end{description}

\section{Design}
\label{chp:study:sec:design}

Genaues vorgehen hier definieren.
High level was ich machen werde und werde dann Metriken berechnen
ursprüngliches Design, dann in exekution beschreiben, was tatsächlich passiert ist.

\section{Study Objects}
\label{chp:study:sec:study_objects}
% Todo Remove list
\begin{enumerate}
    \item Alles was den Datensatz betrifft hier.
    \item MTurk.
    \item Kappa mit Vorversuchen und Drum und dran.
    \item Erst genereller Plan. Dann verschiedene Iterationen (chronologisch).
    \item Resultat → der fertige Datensatz.
    \item Hier die "zweistufigen" Datensatz anführen. Erst nur MTurk, dann Mturk und manuellen.
\end{enumerate}

\section{Execution}
\label{chp:study:sec:execution}
In this section we describe how the study is executed.
The source code used for the study is publicly available \footnote{It can be found at https://github.com/HaaLeo/vague-requirements-scripts}.
Training \ac{ML} models involves the setting of hyperparameter.
Those have a direct impact on the performance of a trained model and the user must set them appropriately to optimize the learning routine and with that the model \parencite{Claesen:2015}.
We are perform a grid search to find a good parameter combination for each model.
We base the parameter grid for each model on the corresponding recommendations of the authors.
Further, we take a re-sampling strategy of our dataset into account for the grid search, because we deal with imbalanced data.
All our models are transformer based.
Transformer based models have the \textit{maximum input sequence length} as hyperparameter.
Input sequences are truncated when they exceed this limit and are padded with 0 if they are shorter than this parameter.
Because of our limited dataset we want to perform $k$-fold cross validation.
In order to determine how many folds we want to use we add this as parameter to our grid as well.
Since all models are transformer based the parameter grids are similar.
The parameter grid for \ac{BERT} and \ac{DistilBERT} is shown in \cref{tab:study:execution:paramter_grid:BERT}.
\begin{table}[htpb]
    \centering
    \begin{tabular}{ p{3.5cm} p{1.5cm} p{2cm} p{1.5cm} p{1.6cm} p{1.5cm} }
        \toprule
         Re-sampling & Folds & Learning Rate & Epochs & Max Length & Batch Size \\
        \midrule
            \begin{itemize}[noitemsep,topsep=0pt,leftmargin=15pt]
                \item {random

                down-sampling}
                \item {random

                up-sampling}
            \end{itemize}
            &\begin{itemize}[noitemsep,topsep=0pt,leftmargin=15pt]
                \item 4
                \item 8
            \end{itemize}
            & \begin{itemize}[noitemsep,topsep=0pt,leftmargin=15pt]
                \item $1\mathrm{e}{-06}$
                \item $5\mathrm{e}{-06}$
                \item $1\mathrm{e}{-05}$
                \item $5\mathrm{e}{-05}$
            \end{itemize}
            & \begin{itemize}[noitemsep,topsep=0pt,leftmargin=15pt]
                \item 1
                \item 2
                \item 3
            \end{itemize}
            & \begin{itemize}[noitemsep,topsep=0pt,leftmargin=15pt]
                \item 64
                \item 128
            \end{itemize}
            & \begin{itemize}[noitemsep,topsep=0pt,leftmargin=15pt]
                \item 16
                \item 32
            \end{itemize}\\
        \bottomrule
    \end{tabular}
    \caption[Parameter Grid for \ac{BERT} and \ac{DistilBERT}]{Parameter grid for \ac{BERT} and \ac{DistilBERT}.}\label{tab:study:execution:paramter_grid:BERT}
\end{table}

The parameter grid for \ac{ERNIE2.0} is similar, however \textcite{Sun:2019a} recommend different learning rates.
This leads to the parameter grid for \ac{ERNIE2.0} shown in \cref{tab:study:execution:paramter_grid:ERNIE2.0}
\begin{table}[htpb]
    \centering
    \begin{tabular}{ p{3.5cm} p{1.5cm} p{2cm} p{1.5cm} p{1.6cm} p{1.5cm} }
        \toprule
         Re-sampling & Folds & Learning Rate & Epochs & Max Length & Batch Size \\
        \midrule
            \begin{itemize}[noitemsep,topsep=0pt,leftmargin=15pt]
                \item {random

                down-sampling}
                \item {random

                up-sampling}
            \end{itemize}
            &\begin{itemize}[noitemsep,topsep=0pt,leftmargin=15pt]
                \item 4
                \item 8
            \end{itemize}
            & \begin{itemize}[noitemsep,topsep=0pt,leftmargin=15pt]
                \item $2\mathrm{e}{-05}$
                \item $3\mathrm{e}{-05}$
                \item $4\mathrm{e}{-05}$
                \item $5\mathrm{e}{-05}$
            \end{itemize}
            & \begin{itemize}[noitemsep,topsep=0pt,leftmargin=15pt]
                \item 1
                \item 2
                \item 3
            \end{itemize}
            & \begin{itemize}[noitemsep,topsep=0pt,leftmargin=15pt]
                \item 64
                \item 128
            \end{itemize}
            & \begin{itemize}[noitemsep,topsep=0pt,leftmargin=15pt]
                \item 16
                \item 32
            \end{itemize}\\
        \bottomrule
    \end{tabular}
    \caption[Parameter Grid for \ac{ERNIE2.0}]{Parameter grid for \ac{ERNIE2.0}.}\label{tab:study:execution:paramter_grid:ERNIE2.0}
\end{table}

Now we perform grid search using the crowdsourced dataset $D_{crowd}$.
For that we split $D_{crowd}$ and used $90\%$ of the data for training and training evaluation whereas the remaining $10\%$ of the data are merely used for the final test of the trained models.
We ensure that the resulting datasets $D_{crowd_{train}}$ and $D_{crowd_{test}}$ consist of the same proportion of vague datapoints of approximately $19\%$.
Finding a suitable learning rate is very challenging task, \textcite{Zeiler:2012} goes even further and states \textit{"[d]etermining a good learning rate becomes more of an art than science for many problems"}.
However, \textcite{Smith:2018} introduced the 1cycle learning rate schedule to dynamically adjust the learning rate during the training phase.
This policy achieved remarkable results in his experiments.
We use this approach and set the initial learning rate to the value of the corresponding parameter grid.
Performing the grid search with the given parameter grids the parameter combinations which achieved the best $F_1$ score are listed in \cref{tab:study:execution:grid_search:results}.
All three models achieved a $F_1$ score around $0.5$.
\begin{table}[htpb]
    \centering
    \begin{tabular}{l p{2.9cm} l p{1.5cm} l p{1.6cm} l }
        \toprule
         Model & Re-sampling & Folds & Learning Rate & Epochs & Max Length & Batch Size \\
        \midrule
        \ac{BERT} & random\newline down-sampling & 4 & $1\mathrm{e}{-05}$ & 2 & 128 &16\\
        \ac{DistilBERT} & random\newline down-sampling & 4 & $5\mathrm{e}{-06}$ & 1 & 64 &32\\
        \ac{ERNIE2.0} & random\newline down-sampling & 4 & $4\mathrm{e}{-05}$ & 2 & 128 &32\\
        \bottomrule
    \end{tabular}
    \caption[Grid Search Results]{The best parameter combinations per model found with grid search.}\label{tab:study:execution:grid_search:results}
\end{table}

% Execution with D_{crowd}
In the first run we train all three models with the best parameter configuration we found with the grid search on $D{crowd_{train}}$.
After that we use the trained models to obtain predictions for the test dataset $D{crowd_{test}}$.
Considering the correctly vague classified requirements as \ac{TP} we then calculate the metrics introduced in \cref{chp:fundamentals:sec:metrics}.
The resulting metrics indicate that the models performed rather poorly on the classification task.
For instance, no model achieves a $F_1$ score greater than $0.6$ and no \ac{AP} greater than $0.5$. %Todo: Precision und recall here?
Instead of listing all results here, we refer to the \hyperref[chp:study:sec:results]{\textit{next section}} where the complete results of all runs are shown comprehensively.

% Why we create D_{manual}
Our trained models perform rather poorly in correctly classifying the requirements of $D{crowd_{test}}$.
According to \textcite{Domingos:2012} a \textit{"dumb algorithm with lots and lots of data [can beat] a clever one with modest amounts of it."}.
It follows, that more data lead to better performing algorithms and \ac{ML} models.
Since we discarded almost the half of all requirements when creating $D_{crowd}$, the concern raises whether our models have had enough data to train properly.
To encounter this concern and to include all available requirements in the dataset we relabeled all discarded requirements like we described in \cref{chp:study:sec:study_objects:dataset_creation:manual_labeling}.

% Execution with D_{all}
For the second run we now use the more extensive dataset $D_{all}$.
We again split it in a training partition $D_{all_{train}}$ and test dataset $D_{all_{test}}$ with the proportion $1\mathrm{:}9$ preserving the distribution of vague requirements for both.
We then train all of our models on the new dataset $D_{all_{train}}$ with the hyperparameters obtained by the preceding grid search.
Analogically to the first run, we obtain predictions for the dataset $D_{all_{test}}$ and then calculated metrics for the evaluation considering the correctly as vague classified requirements as \ac{TP}.

\section{Results}
\label{chp:study:sec:results}

\section{Interpretation}
\label{chp:study:sec:interpretation}

% Introduction paragraph. why are results bad
When we compare our results from \cref{chp:study:sec:results} with more \hyperref[chp:related_research:sec:rule_based approaches]{\textit{traditional approaches}}, one immediately observes that our models perform rather poorly.
One example is Smella, developed by \textcite{Femmer:2017}.
It uncovers \textit{requirement smells} with a precision of $59\%$ and a $82\%$ whereas our \ac{BERT} based model achieves same recall, but only a precision of $0.36$.
In contrast to our \textit{vague requirements} Smella's objective is to identify requirement smells.
Although these objectives are not identical, they are very similar.
Therefore, we can follow that Smella as an example and rule based approaches in general are severely better in identifying poor requirements.
In this section we want to reflect on the previous results presented and discuss possible causes.

% Paragraph for LIME
First we want to check which word of a sentence contributes most to the model's classification result.
To do this we use \ac{LIME}.
After that we count the occurrences of those words in $D_{all_{train}}$ and try to derive \textit{why} our models classify a majority of the dataset wrongly.
With this procedure we analyze the top four \ac{TP}, \ac{TN}, \ac{FP} and \ac{FN} of a model.
An example for a false positive analyzed by \ac{LIME} is given in \cref{fig:study:interpretation:LIME}.
The more a word's background is highlighted in green, the more it contributes to the overall prediction, in case of \cref{fig:study:interpretation:LIME} the prediction is \textit{vague}.
\begin{figure}[htpb]
    \centering
    \def\svgwidth{\columnwidth}
    \scalebox{0.80}{\subsection{Local Interpretable Model-Agnostic Explanations}
\label{chp:fundamentals:sec:machine_learning:subsec:transfer_learning}
According to \textcite{Ribeiro:2016} \ac{ML} models are widely spread, although they mostly remain black boxes.
Further, they state that is essential to be able to trust a model when it used to make decisions.
However, it is very difficult for a human to trust a model which cannot explain its predictions.
To address this issue \textcite{Ribeiro:2016} introduce \ac{LIME}.

When one wants to apply \ac{LIME} for a sample $s$ to obtain an explanation for the prediction it first samples datapoints near $s$ and weights them with their distance to $s$.
Then the predictions for the sampled datapoints are generated by the original model.
This labeled dataset is used to train an interpretable linear model which approximates the original well near $s$.
Consequently, \ac{LIME} presumes that a complex model is linear on a local scale.
The explanations for the prediction of $s$ of the original model are then derived using the interpretable model. \parencite{Ribeiro:2016}
}
    \caption[Study Interpretation: Example for LIME]{LIME scores for a false positive.}\label{fig:study:interpretation:LIME}
\end{figure}

\newpage % Todo check page break
After we we determined the occurrences of the requirements' words in the dataset we could not identify a pattern among how often a word occurs in the dataset and how it influences the prediction.
How often the words of above figure occur in vague and not-vague requirements in $D_{all_{train}}$ is shown in \cref{tab:study:interpretation:LIME}.
\begin{table}[htpb]
    \centering
    \begin{tabular}{l l l l l }
        \toprule
         Word & \ac{LIME} Score & Count in Vague Req. & Count in Not Vague Req. \\
        \midrule
        analyzes & 0.164 & 6 & 3 \\
        remaining & 0.201 & 1 & 3  \\
        life &  0.149 & 13 & 14\\
        inspection & 0.05 & 3 & 14 \\
        intervals & 0.063 & 2 & 2 \\
        \bottomrule
    \end{tabular}
    \caption[Study Interpretation: Word Occurrences]{The occurrences of a requirement's words in $D_{all_{train}}$.}\label{tab:study:interpretation:LIME}
\end{table}
% Paragraph for Architecture/ Self Attention

% Paragraph on why we do not use thresholds to "enhance" results.


\chapter{Threats to Validity}
\label{chp:threats_to_validity}
In the prior chapters we introduced a transformer based approach to classify requirements as vague or not-vague.
We used it to conduct a study whose result is that transformer based models must be further enhanced to be used for reliable requirement predictions.
The approaches performed rather poorly.
Therefore, in this chapter we want to elaborate possible threats to the study's validity.

% Paragraph on dataset quality
The first valid concern raising is the datasets' quality.
Although we define a quality condition to ensure a dataset's quality which is quite intuitive, the condition is arbitrary similar to the thresholds defined by \textcite{Landis:1977}.
This means that that even our crowdsourced dataset meets the condition it is questionable whether the condition itself is sufficient for a high quality dataset.
However, this concern is only valid for a part of our dataset.
The dataset $D_{manual}$ is excluded from this threat, because it was labeled by an expert.

% Paragraph on Implementation Mistakes
For the study execution and its evaluation we have written hundreds lines of code.
The code was manually tested as well as by some unit tests.
The implementation follows the study design and we removed all flaws which we found.
Nevertheless, there remains a small chance that the implemented program still contains mistakes made either by the author or included in third party software libraries.
Since we cannot guarantee that all components of our program are flawless, this is a valid threat to the validity.

% Paragraph on bad grid search
The hyperparameters used for the training pose another threat.
Finding suitable hyperparameters is very challenging \parencite{Zeiler:2012} and has become its own field of research.
In our study we aim to determine good hyperparameters with a rather simple grid search.
With this technique one can detect the best parameter configuration of the earlier defined parameter, but we do not know how good the parameter grid itself is.
Although we created a parameter grid based on the recommendations of the models' authors, it is likely that we obtained not optimal hyperparameters which can lead to worse training and consequently to a less performant models.
Therefore, the non optimal hyperparameters are a threat as well.

% Paragraph on "to easy" downstream NN
The last threat concerns the approach.
All three models output multiple tokens whose first one can be used for downstream classification tasks \parencite{Devlin:2018}.
For this downstream classification we used a fully connected feed forward \ac{NN} with around 1500 learnable parameters.
Using a low-parameter \ac{NN} harbors the the risk that it cannot map the vagueness' complexity to the correct prediction.
Since we cannot preclude that the downstream \ac{NN} is incapable for this specific classification task, this is considered as a threat to validity.

\chapter{Relation to Existing Evidence}
\label{chp:relation_to_existing_evidence}

\input{chapters/00_test/00_test}

\appendix{}

\microtypesetup{protrusion=false}
\listoffigures{}
\listoftables{}
\microtypesetup{protrusion=true}

\printacronyms[name=List of Abbreviations and Acronyms]

\printbibliography{}

\end{document}
